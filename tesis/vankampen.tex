\subsection{Teorema de \vank}
El teorema de \vank nos permite calcular el grupo fundamental de un
espacio que luzca como una \emph{wedge sum} de espacios puntuados

\begin{teorema}
  Sea \(X = U \cup V\), donde \(U,V\) son conjuntos abiertos de \(X\).
  Supongamos que \(U \cap V\) es arco-conexo y que \(x_0 \in U \cap V\).
  Sea \(i\) y \(j\) las inclusiones de \(U\) y \(V\) en \(X\)
  respectivamente. Entonces las imagenes de los homomorfismos inducidos
  \[ i_* : \pi (U, x_0) \to \pi (X, x_0) \quad j_* : \pi (V, x_0) \to
  \pi (X, x_0) \]
  generan a \(\pi (X,x_0)\)
\end{teorema}
\begin{proof}
  Hemos de probar que para todo arco \(f\) en \(X\) basado en \(x_0\),
  es arco-homotopico a un arco de la forma
  \[ g_1 * g_2 * ... * g_n \]
  donde cada \(g_i\) es un arco en \(X\) basado en \(x_0\) que esta
  contenido completamente en \(U\) o en \(V\).

  Para esto, mostramos que existe una subdivision de \(I\) dada por
  \(a_0 < a_1 < ... < a_n \) tal que \(f(a_i) \in U \cap V\) y que \(
  f([a_i , a_{i+1}]) \) es subconjunto contenido solo en \(U\) o en
  \(V\).

  Por lema \ref{thm:lebesgue-number-lema} (numero de Lebesgue),
  existe una subdivision de \(I\) dada por \(b_1 < ... < b_n\) tal que
  \(f ([b_i , b_{i+1}])\) pertenece completamente a \(U\) o \(V\) para
  todo \(i \in [1,n]\). Si para para todo \(i\) se cumpliera que
  \(f(b_i) \in U \cap V\) hemos terminado. En caso contrario, de existir
  \(i\) tal que \(f(b_i) \not \in U \cap V\) esto implica que
  \[ f(b_i) \in U \ \veebar \ f(b_i) \in V \]
  Tomando s.p.d.g la primera alternativa, implicaria que
  \[ f(b_i) \in U \implies f([b_{i-1}, b_{i}]) \subset U \ \land \ f([b_i,
    b_{i+1}]) \subset U \]
  Analogamente si \(f(b_i) \in V\). Podemos olvidar entonces el valor
  \(b_i\) de la subdivision y revisar si \(f(b_{i+1}) \in U \cap V\),
  repitiendo este proceso una cantidad finita de veces hasta satisfacer
  la condicion.

  Ahora se prueba el teorema en si. Dado \(f\) y \(a_0 < ... < a_n\)
  subdivision anterior, definamos las funciones
  \[ l_i : [0,1] \to [a_{i-1}, a_{i}] \]
  \[ f_i = f \circ l_i \]
  Es claro que \(\forall i \in [1, n], \ f_i (I) \) esta contenida en
  \(U\) o en \(V\). Tambien es claro por simple calculo que se cumple
  que
  \[ [f] = [f_1] * [f_2] * ... * [f_n] \]
  Pero estos \(\{f_i\}\) no son arcos cerrados basados en \(x_0\) de
  \(\pi (U,x_0)\) o \(\pi (V,x_0)\). Para solucionar esto, nos basamos
  en la arco conexidad de \(U \cap V\), lo que nos permite afirmar que
  para todo \(i \in [1 , n]\) existe
  \[ \alpha_i : I \to U \cap V \]
  \[ \alpha_i (0) = x_0 \quad \alpha_i (1) = f(a_i) \]
  arcos continuos, fijando ademas \(\alpha_0 \equiv x_0 \equiv \alpha_n
  \). Estos nos permiten construir los arcos
  \[ g_i = \alpha_{i-1} * f_i * \alpha_i^{-1} \]
  los cuales si estan basados en \(x_0\), y cuya imagen sigue estando
  contenida en \(U\) o en \(V\). Por simple calculo se ve que
  \[ [g_1] * ... * [g_n] = [f_1] * ... * [f_n] = [f] \]
\end{proof}
