\section{Categoría de homotopias}
En la sección anterior, se termino dando una definición de categoría con
\(\mathscr{Top}_*\) y el functor \(_{{*}} : \mathscr{Top} \to
\mathscr{Grp}\). Los morfismos en \(\mathscr{Top}_*\) correspondían a
homeomorfismos entre espacios topológicos, los cuales inducían bajo
\(_{(*)}\) isomorfismos de grupos. Pero hay espacios que a pesar de no
ser homeomorfos poseen el mismo grupo fundamental y para estos el
functor entre \(\mathscr{Top}_* \to \mathscr{Grp}\) es ciego a su
estructura. Esto es insatisfactorio pensando en la meta original de
clasificar diferentes espacio topológicos y nos hace pensar en que
tal vez la noción de homeomorfismos es mucho mas fuerte de lo que
necesitamos.

Entre las nociones alternativas mas populares se encuentra las
\emph{equivalencias homotopicas}. Su popularidad se debe a que son mas
débiles que un homeomorfismo y que corresponden a los morfismos en una
nueva estructura categórica denotada por \(\mathscr{HoTop}_*\), la cual
contiene como sub-categoría a \(Top_*\). La motivación de su
construcción es natural a partir del estudio de retracciones, las cuales
estudiaremos y iremos generalizando debidamente.

\subsection{Retracciones}
Iniciaremos estudiando un pequeño lema tecnico con respecto a la
composicion y la funcion identidad.
\begin{lema}
  Sea \(f : X \to Y\) e \(g : Y \to X\) dos funciones. Si \( g \circ f =
  h : X \to X \), con \(h\) una biyeccion, entonces \(f\) es inyectiva y
  \(g\) es sobreyectiva.
\end{lema}
\begin{proof}
  Se argumenta por contradiccion. Supongamos que \(f\) no es inyectiva o
  que \(g\) no es sobreyectiva. Tomando el primer caso para \(f\) se
  tiene que existen \(x_1 , x_2 \in X,\ x_1 \neq x_2\) que cumplen
  \[ f (x_1) = f(x_2) \]
  Dado que \(g\) es una funcion, debe de cumplirse
  \[ g (f (x_1)) = g (f(x_2)) \]
  \[ h (x_1) = h(x_2) \]
  Lo cual es una contradiccion con la biyectividad de \(h\).

  Por otro lado, suponiendo que \(g\) no es sobreyectiva, deberia de
  existir \(x \in X\) tal que no exista \( y \in Y\) que \(g (y) = x\).
  Pero por hipotesis, el elemento \(f(h^{-1}(x)) \in Y\) es tal que \(g
  (f (h^{-1}(x))) = x\). Mostrando asi que \(g\) es en efecto
  sobreyectiva.
\end{proof}
\begin{corolario} \label{thm:comp-identidad}
  Sea \(f : X \to Y\) e \(g : Y \to X\) dos funciones continuas. Si \( g
  \circ f = Id : X \to X \), entonces \(f\) es inyectiva y \(g\) es
sobreyectiva.
\end{corolario}
\noindent Con esto en mente, podemos pasar directamente a la definicion
de retracion.
\begin{definicion}
  Sea \(X\) un espacio topologico. \(A \subset X\) es una retraccion de
  \(X\) si existe un mapeo continuo \(r : X \to A\) tal que
  \[ r \mid_{A} (x) = x \]
  En tal caso \(r\) es llamada la aplicacion retraccion de \(X\) en \(A\).
\end{definicion}
Adicionalmente podemos definir trivialmente una inclusion \(j : A \to
X\). Con estas functiones tenemos el siguiente teorema
\begin{teorema} \label{thm:retraccion-inclusion}
Si \(A \subset X\) es una retraccion, entonces la inclusion \(j : A \to
X\) induce un homomorfismo de grupos fundamentales \(j_{*} : \pi(A, a)
\to \pi(X,a)\) inyectivo.
\end{teorema}
\begin{proof}
  La composicion \(r \circ j : A \to A\) es la funcion identidad de
  \(A\), por el corolario \ref{thm:comp-identidad}, sabemos que \(j\)
  debe de ser inyectivo. Su homorfismo inducido
  \[ (r \circ j)_{*} = r_{*} \circ j_{*} \]
  es el homorfismo identidad entre \(\pi(A,a) \to \pi(A,a)\). Otra vez
  por el corolario \ref{thm:comp-identidad}, esto implica que \(r_{*}\)
  es una sobreyeccion y que \(j_{*}\) es inyectiva.
\end{proof}
La retraccion entonces nos da un embedimiento del grupo fundamental de
\( A \) en \(X\). Ya podemos ver un poco de las consecuencias de este
resultado, por ejemplo diciendonos gracias a su contrapositivo que no
existe una retracion de \(B^2\) en \(S^1\), ya que si lo hubiera, la
inclusion \(j : S^1 \to B^2\) seria inyectiva, pero el grupo fundamental
de \(B^2\) es trivial y el de \(S^1\) no lo es (corresponde a \((\mathbb
Z, +)\)). Esto ultimo es un resultado que veremos mas adelante
mediante espacios cubrimientos. Esta ultima afirmacion, nos permite
completar lo afirmado en cuanto al grupo fundamental de \(\mathbb R ^2 -
\{(1,1)\}\).

\begin{corolario}
  Para todo \(\vec a \in \mathbb R ^2\), el grupo fundamental de \(\mathbb R ^2 - \{\vec a\}\) es \((\mathbb Z, +)\).
\end{corolario}
\begin{proof}
Dado que para todo \(\vec a \in \mathbb R ^2 \), siempre existe
el homoemorfismo
\[ f_a : \mathbb R ^2 - \{\vec a\} \to \mathbb R ^2 - \{\vec 0\} \]
\[ \vec x \mapsto \vec x - \vec a \]
gracias a este, podemos estudiar el grupo fundamental de \(\mathbb R ^2
- \{\vec 0\}\) sin perdida de generalidad mediante el teorema
\ref{thm:homoemorfismo-isomorfismo}.

Para este, podemos construir la siguiente retraccion a \(S^1\)
\[ r : \mathbb R ^2 - \{\vec 0\} \to S^1 \]
\[ x \mapsto \frac x {\lVert x \rVert} \]
Esta cumple la condicionn de continuidad trivialmente y dado que todo
elemento de \(S^1\) posee norma \(1\), para estos corresponde a la
identidad, como es requerido para una retraccion. Luego, por teorema
\ref{thm:retraccion-inclusion} tenemos que
\[ j_* : \pi (S^1 , y_0) \to \pi (\mathbb R ^2 - \{\vec a\}, x_0) \]
es una inyeccion de grupos fundamentales

\end{proof}

Entre retracciones que son conocidas, estan \(\mathbb R ^2 - \{0\}\) en
\(S^1\) mediante la funcion \(r (x) = x / \lVert x \rVert \). Lo que nos
dice por el teorema anterior que \(j_{*} : \pi (S^1, a) \to \pi (\mathbb
R ^2 - \{0\}, a)\) es inyectiva. Si mediante algun resultado pudieramos
probar que \(j_{*}\) es ademas sobreyectiva tendriamos listo nuestro
isomorfismo de grupo. Esto es posible de construir pero para eso
necesitamos algunos resultados tecnicos.

\begin{lema} \label{lem:homotopic-inducing}
  Sean \(h,k : (X, x_0) \to (Y, y_0)\) dos mapeos continuous. Si \(h\) y
  \(k\) son homotopicos y si la imagen de \(x_0\) bajo este permanece
  fija en \(y_0\) durante la homotopia, entonces \(h_*\) e \(k_*\) son iguales.
\end{lema}
% TODO: hay que ser mas preciso porque debemos mantener la homotopia
% fija en un punto. Hay que hacer enfasis en el grupo fundamental
% basandose en lo fijo de dicho punto.
\begin{proof}
  Queremos mostrar que \(\forall [f] \in \pi (X,x_0)\), se cumple que
  \(h_* ([f]) = k_* ([f])\). Esto equivale a mostrar que
  \[ [h \circ f] = [k \circ f] \]
  Es decir, debemos de encontrar una homotopia entre \((h \circ f)\) e \(
  (k \circ f)\).

  Para esto, usamos la homotopia \(H : X \times I \to Y \) entre \(h\) y
  \(k\). Notando que \(f : I \to X\) podemos construir una homotopia \(M
  : I \times I \to Y \) pre-componiendo como
  \[ (z, t) \mapsto H \circ (f(z), t) \]
  La cual es una homotopia entre \((h \circ f)\) e \((k \circ f)\)

  La importancia de que se cumpla que \( \forall t \in I,\ M (0, t) =
  y_0\) es para decir que los espacios de llegada son los mismos
  espacios puntuados \((Y,y_0)\). Por tanto \(h_* = k_*\).
\end{proof}
\begin{teorema} \label{thm:comp-identidad-homotopia}
  Sea \(f : X \to Y\) y \(g : Y \to X\) funciones continuas. Si \(f
  \circ g\) es homotopico a la identidad \( Id : Y \to Y\) y existe
  \(y_0 \in Y\) tal que \( (f \circ g ) (y_0) = y_0 \) entonces
  \(f_*\) es sobreyectivo y \(g_*\) es inyectivo.
\end{teorema}
\begin{proof}
  Aplicando el lema \ref{lem:homotopic-inducing} sobre \(f \circ g \) y
  \( Id : Y \to Y\), se obtiene la igualdad
  \[ (f \circ g)_{*} = f_* \circ g_* = Id_* \]
  Donde \(Id_* : \pi (Y, y_0) \to \pi (Y, y_0)\). Luego aplicando el corolario
  \ref{thm:comp-identidad} sobre esta ecuacion obtenemos que \(f_*\) es
  sobreyectiva y que \(g_*\) es inyectiva.
\end{proof}
% \begin{proof}
%   Se procede de manera analoga al teorema \ref{thm:comp-identidad}.
%   Supongamos que \(g_*\) no es inyectiva, es decir existen \([a],[b] \in
%   \pi (Y,y_0), [a] \neq [b]\) tales que
%   \[ g_* ([a]) = g_* ([b]) \]
%   Dado que \(f_*\) es una funcion y por tanto solo depende sus
%   argumentos, aplicando a ambos lados \(f_*\)
%   \[ f_* (g_* ([a])) = f_* (g_* ([b])) \]
%   \[ [a] = [b] \]
%   Lo cual es una contradiccion con la suposicion inicial.

%   De manera analoga, para mostrar que \(f_*\) es sobreyectiva,
%   supongamos que existe \([c] \in \pi (Y, y_0)\) tal que no existe \([d]
%   \in \pi (X,x_0)\) tal que \(g_* ([d]) = [c]\). Pero por cumplirse que
%   \( f_* \circ g_* = Id_* : \pi (Y, y_0) \to \pi (Y, y_0)\), sabemos que
%   existe \(g_* ([c])\) que cumple dicha relacion, por tanto \(f_*\) es
%   sobreyectiva.
% \end{proof}
Esta es exactamente la misma demostracion que en el teorema
\ref{thm:comp-identidad}, pero haciendo enfasis lo que significa ser
inyectivo o sobreyectivo en clases de equivalencias. Resulta que ya
tenemos todos los resultados necesarios para expandir el resultado de la
retraccion \(\mathbb R ^2 - \{0\}\) en \(S^1\).

\begin{teorema}
  Para todo \(a \in S^1\), la inclusion homomorfica \(j_* : \pi (S^1, a)
  \to \pi (\mathbb R ^2 - \{0\}, a)\) es un isomorfismo de grupos
  fundamentales.
\end{teorema}
\begin{proof}
  Ya habiamos dicho anteriormente que \( j_* : \pi (S^1, a) \to \pi
  (\mathbb R ^2 - \{0\}, a)\) es inyectiva como homomorfismo de grupos
  fundamentales, puesto la existencia de la retraccion
  \begin{align*}
    r : \mathbb R ^2 - \{0\} &\to S^1 \\
    x &\mapsto \frac x {\lVert x \rVert}
  \end{align*}
  implica por corolario \ref{thm:comp-identidad} la inyectividad.

  Para probar la sobreyectividad de \(j_*\), tomemos la composicion
  \[ j \circ r : \mathbb R ^2 - \{0\} \to \mathbb R ^2 - \{0\} \]
  Esta composicion \emph{no} es la identidad de \(\mathbb R ^2 - \{0\}
  \) pero es homotopica a esta mediante la homotopia de linea recta
  \[ H(x,t) := t \cdot x + (1 - t) \cdot \frac x {\lVert x \rVert} \]
  Esta homotopia deja fijo al punto \((1,0)\), luego tenemos las
  hipotesis del teorema \ref{thm:comp-identidad-homotopia} y obtenemos
  que \(j_*\) es ademas sobreyectiva. Concluimos que \(j_*\) es un
  isomorfismo de grupos fundamentales.
\end{proof}

\subsection{Tipos homotopicos}
Ahora que ya conocemos el esquema de trabajo, podemos generalizar para
espacios que no necesariamente sean retracciones entre si, pero que
posean un par de funciones cuyas composiciones sean homotopicas a la
identidad correspondiente.
\begin{definicion}
  Sean \(f : X \to Y\) e \(g : Y \to X\) mapeos continuos. Supongamos
  que \( g \circ f : X \to X \) es homotopico al mapeo identidad de
  \(X\) y \( f \circ g : Y \to Y \) es homotopico al mapeo identidad de
  \(Y\). Entonces \(f\) y \(g\) son llamadas \emph{equivalencias
  homotopicas} y cada una es una \emph{inversa homotopica} de la otra.
  Si dos espacios poseen un par de equivalencias homotopicas entre ellos,
  se dicen que ambos espacios son del mismo \emph{tipo homotopico}.
\end{definicion}
Podemos ver que en el caso anterior, la retraccion y la inclusion
correspondian a par de equivalencias homotopicas. Podria pensarse que la
existencia de un par de equivalencias homotopicas entre espacios seria
suficiente para afirmar que poseen el mismo grupo fundamental, esto es
cierto en medida que se sea cuidadoso con los puntos de partida. Este
problema no se manifestaba en retracciones porque el punto de partida se
mantenia fijo en la inclusion. Para ver el problema mas claro,
supongamos que tenemos dos espacios \(X, Y\) junto con un par
equivalencias homotopicas
\[ f : X \to Y \quad g : Y \to X \]
Por hipotesis tenemos que \(g \circ f \simeq Id : X \to X\), pero eso
\textbf{no} nos asegura la existencia de un punto \(x_0 \in X\) que sea
fijo bajo la homotopia entre \( g \circ f\) y \(Id\). Por tanto no
podemos aplicar el lema \ref{lem:homotopic-inducing} pues no podemos
cumplir una de las hipotesis. Resulta que esto es subsanable utilizando
el siguiente lema tecnico.

\begin{lema} \label{lem:equiv-hom-lift}
  Sean \(h,k : X \to Y\) mapeos continuos, sea \(h (x_0) = y_0,\ k(x_0)
  = y_1\). Si \(h\) y \(k\) son homotopicos mediante \(H : X \times I
  \to Y\), entonces existe un arco \(\alpha\) en \(Y\) de \(y_0\) a
  \(y_1\) tal que \(k_* = \hat \alpha \circ h_* \). Mas aun, este arco
  esta dado por la ecuacion \(\alpha (t) = H (x_0, t)\).
\end{lema}
\noindent Aqui se utilizara la notacion \(\hat \alpha\) que se utilizo
pero no definicion en la demostracion de el teorema \ref{not:alpha-hat}.
\begin{proof}
  Hemos de probar que \(\forall [f] \in \pi (X, x_0)\) se cumple que
  \[ k_* ([f]) = \hat \alpha \circ h_* ([f]) \]
  \[ [k \circ f] = [ \alpha^{-1} ] *  [h \circ f] * [\alpha] \]
  \[ [ \alpha ] * [k \circ f] =  [h \circ f] * [\alpha] \]
  Es decir, hemos de construir una homotopia entre estos dos
  representantes de clase. Sea \(H : X \times I \to Y\) la homotopia
  entre \(h\) y \(k\) dada por hipotesis, entonces podemos reescribir
  las funciones de la ultima ecuacion definiendo nuevas curvas sobre \(I
  \times I\)
  \[ f_0(t) := (f(t), 0) \quad H \circ f_0 = h \circ f \]
  \[ f_1(t) := (f(t), 1) \quad H \circ f_1 = k \circ f \]
  \[ c(t) := (x_0, t) \quad H \circ c = \alpha \]

  De igual manera re-definiremos las tres nuevas curvas en terminos de
  \(F : I \times I \to X \times I\) dada por la ecuacion \(F(s,t) =
  (f(s),t)\). Para esto debemos considerar los cuatro bordes de
  \(I \times I\).
  \[ \beta_0(s) = (s, 0) \quad \beta_1(s) = (s, 1)\]
  \[ \gamma_0(t) = (0, t) \quad \gamma_1(t) = (1, t)\]
  Mostrandose las siguientes igualdades
  \[ F \circ \beta_0 = f_0 \quad F \circ \beta_1 = f_1\]
  \[ F \circ \gamma_0 = c = F \circ \gamma_1 \]

  Notese que los caminos \(\beta_0 * \gamma_1\) y \(\gamma_0 * \beta_1\)
  son los caminos inferior derecho y izquierdo superior del cuadrilatero
  \(I \times I\). Dada la convexidad de este espacio, existe una
  homotopia denotada por \(G : I \times I \to I \times I \) entre ellos.
  Luego por composicion \(F \circ G\) es una homotopia entre \(f_0 * c\) y
  \(c * f_1\) y la composicion \(H \circ F \circ G\) es una homotopia en
  \(Y\) tal que
  \[ (H \circ f_0) * (H \circ c) = (h \circ f) * \alpha \]
  \[ (H \circ c) * (H \circ f_1) = \alpha * (k \circ f) \]
  Obteniendo asi la homotopia buscada.
\end{proof}

La importancia de encontrar dicho arco \(\alpha : I \to Y\) entre los
dos levantamientos proviene del siguiente lema.
\begin{lema} \label{lem:hat-alpha-iso}
\(\hat \alpha : \pi (Y, y_0) \to \pi (Y, y_1)\) es un isomorfismo de
grupos fundamentales.
\end{lema}
\begin{proof}
  Por definicion \(\hat \alpha\) es claramente un homomorfismo. Para ver
  que es inyectiva, supongamos que tomamos \([f],[g] \in \pi (Y, y_0),\
  [f] \neq [g]\) tal que se cumple \(\hat \alpha ([f]) = \hat \alpha
  ([g])\). Esto implicaria por reduccion que
  \[ [\alpha^{-1}] * [f] * [\alpha] = [\alpha^{-1}] * [g] * [\alpha] \]
  \[ [f] * [\alpha] = [g] * [\alpha] \]
  \[ [f] = [g] \]
  Lo cual es una contradiccion

  Para ver la sobreyectividad, para todo elemento \([h]
  \in \pi (Y,y_0)\) existe \( [\alpha] * [h] * [\alpha^{-1}] \in \pi
  (Y,y_0)\) tal que
  \[ \hat \alpha ([\alpha] * [h] * [\alpha^{-1}]) = [h]\]
\end{proof}
Con esto es mente, es facil ver que se tiene el siguiente corolario
\begin{corolario}
  Sean \(h,k : X \to Y\) homotopicamente continuas tales que \(h (x_0) =
  y_0\) y \(k(x_0) = y_1\). Si \(h_*\) es inyectiva o sobreyectiva o
  trivial, entonces \(k_*\) tambien lo es.
\end{corolario}
\begin{proof}
  Dado que por el lema \ref{lem:equiv-hom-lift} tenemos la siguiente
  ecuacion entre levantamientos
  \[ k_* = \hat \alpha \circ h_* \]
  Por el lema \ref{lem:hat-alpha-iso} sabemos que \(\hat \alpha\) es
  biyectivo. Por tanto la composicion de \(h_*\) inyectiva con \(\hat
  \alpha\) bijectiva es inyectiva. De igual manera para sobreyectividad y
  la trivialidad.
\end{proof}
Con esto podemos al fin plantear que todo par de espacios en el mismo
tipo homotopico poseen el mismo grupo fundamental. Esto gracias a que
podemos evitar la condicion de punto fijo en el lema
\ref{lem:homotopic-inducing} gracias a la biyeccion de \(\hat \alpha\).

\begin{teorema}
  Sean \(f : X \to Y\) continua y \(f (x_0) = y_0\). Si \(f\) es una
  homotopia equivalencia con \(g : Y \to X\), entonces
  \[ f_* : \pi (X, x_0) \to \pi (Y, y_0)\]
  es un isomorfismo.
\end{teorema}
Para este teorema, es necesario que seamos un poco mas especificos con la
notacion de levantamientos. Un levantamiento de \(f :
X \to Y\) con respecto a \(x_0\) corresponde a la funcion
\[ (f_{x_0})_{*} : \pi (X , x_0) \to \pi (Y, y_0)\]
donde \(f(x_0) = y_0\). Ahora podemos proceder con la demostracion.
\begin{proof}
  Fijemos los puntos \(x_1 = g(y_0)\) y \(y_1 = f(x_1)\). Vease que dado
  los diferentes puntos bases se tienen dos homomorfismos inducidos de
  \(f\) con distintos puntos relativos \[(f_{x_0})_* : \pi (X, x_0) \to \pi
  (Y, y_0)\] \[(f_{x_1})_* : \pi (X, x_1) \to \pi (Y, y_1)\].

  Por hipotesis, \(g \circ f : (X, x_0) \to (X, x_1)\) es homotopico a
  la funcion identidad \(Id : X \to X\). Luego por el lema
  \ref{lem:equiv-hom-lift} se tiene que
  \begin{equation} \label{eq:alpha-f-g}
  g_* \circ (f_{x_0})_* = (g \circ f)_* = \hat \alpha \circ (Id_X)_*
    = \hat \alpha
  \end{equation}
  Por tanto \(g_* \circ (f_{x_0})_*\) es un isomorfismo y esto obliga a
  que \(g_*\) sea sobreyectivo.

  De igual manera, aplicando el analisis anterior a \(f \circ g : \pi
  (Y, y_0) \to (Y, y_1)\) homotopico a la identidad de \(Y\), se
  obtiene que \((f_{x_1})_* \circ g_*\) es un isomorfismo y por tanto
  \(g_*\) es inyectivo tambien.

  Conociendo la biyectividad de \(g_*\), nos permite pre-componer por su
  inversa \((g_*)^{-1}\) en la ecuacion \eqref{eq:alpha-f-g}, obteniendo
  \[ (f_{x_0})_* = (g_*)^{-1} \circ \hat \alpha\]
  luego, la composicion de biyeccion es biyeccion y por tanto
  \((f_{x_0})_*\) es un isomorfismo de grupos fundamentales.
\end{proof}

Con estos resultados podemos ver el siguiente ejemplo
\begin{corolario}
  Sea \(M\) la banda de \emph{Mobius} finita definida por la identificacion
  \[ M := [0,1] \times [0,1] / \sim \quad (0,y) \sim (1, 1 - y) \]
  Este espacio posee el mismo grupo fundamental de \(S^1\)
\end{corolario}
\begin{proof}
  %%TODO: diagama de M
  Podemos pensar en la banda de \emph{Mobius} como un cuadrado con los
  lados izquierdos y derechos identificados en sentido inverso.
  Definimos la ``banda central'' de este como el conjunto
  \[ D := [0,1] \times \{\frac 1 2\} / (0, 1 / 2) \sim (1,1 / 2) \]
  (esta identificacion equivale a la de \(M\) para \(y = 1/2\)). Podemos
  definir la siguiente retraccion e inclusion
  \[
    \begin{matrix}
      r : M \to D & j : D \to M \\
      [(x,y)] \mapsto [(x, 1/2)] & x \mapsto x
    \end{matrix}
  \]
  Hemos de probar que \(j \circ r\) es homotopico a la identidad \(Id
  : M \to M\). Definamos la homotopia
  \[ F : M \times I \to M \]
  \[ ([(x,y)], t) \mapsto [(x, \frac t 2 + (1-t) y)] \]
  La cual es claramente continua y cumple que \( F(.,0) = Id,\ F(.,1) =
  j \circ r\). Luego por teoremas anterior, el grupo fundamental de
  \(M\) es igual al de \(D\).

  La pregunta ahora es ver quien es el grupo fundamental de \(D\).
  Podemos ver que este conjunto es una recta con los dos extremos
  identificados, lo cual luce similar a \(S^1\). Estos espacios son
  homeomorfos mediantes la aplicacion \[[(x, 1/2)] \mapsto \exp (2 \pi
  x)\], mostrando asi el resultado deseado.

\end{proof}
\subsection{La categoria \(\mathscr{HoTop}_*\)}
\begin{definicion}
  La categoria \(\mathscr{HoTop}_*\) corresponde a una tripleta
  \((\mathbf{O},\mathbf M, \circ)\) donde
  \begin{itemize}
  \item \(\mathbf O\) corresponde a espacio topologicos puntuados de la
    forma \((X, x_0)\).
  \item \(\mathbf M\) corresponde a un conjunto
    \[ M = \{ [f] \mid f : (X,x_0) \to (Y,y_0),\ (X,x_0),(Y,y_0) \in
      \mathbf O \}\]
    de clases equivalencias homotopicas entre funciones continuas
  \item \((\circ)\) corresponde a la composicion de clases de
    equivalencias tal que si \([f] , [g] \in \mathbf O \) tal que
    \[ f : (X,x_0) \to (Y, y_0)\]
    \[ g : (Y, y_0) \to (Z, z_0)\]
    entonces \([g] \circ [f]\) es una clase de equivalencia de \((X,
    x_0) \to (Z, z_0)\)
  \end{itemize}
\end{definicion}
Esta categoria es diferente a \(\mathscr{Top}_*\) en cuanto a sus
morfismos, pero es posible incrustar mediante un functor
\begin{align*}
  F : \mathscr{Top}_* &\to \mathscr{HoTop}_* \\
      f &\mapsto [f]
\end{align*}
en particular, dado que los morfismos de  \(\mathscr{Top}_*\) son
homeomorfismos, estos son trivialmente equivalencias homotopicas,
dandonos \emph{equivalencias categoricas} (presencia de inversas) en
en  \(\mathscr{HoTop}_*\).

Algo diferente en esta construccion es la presencia de clases de
equivalencias entre funciones. Esto tiene que ver con el requerimiento
de unicidad sobre el morfismo identidad. Esto tiene la consecuencia
feliz de simplificar la visualizacion de esta categoria como diagrama.

Esta estructura nos ayudara a deducir grupos fundamentales de espacios
complejos en base a espacios mas simples. Una de las herramientas para
esto sera el estudio de los productos y co-productos categoricos.

\begin{definicion}
  Sea \(P , X , Y \in \mathbf O\) objetos arbitrarios de un categoria
  tal que existen \(\pi_x : P \to X,\ \pi_y : P \to Y\) morfismos en
  \(\mathbf M\). Si para todo objeto \(M \in \mathbf O\) tal que posea
  morfimos \( f : M \to X,\ g : M \to Y\) existe un morfismo \(\theta :
  M \to P\) tal que
  \[ f = \pi_x \circ \theta \quad g = \pi_y \circ \theta \]
  Entonces \(P\) es un producto de \(X\) e \(Y\).
\end{definicion}
Analogamente, un co-producto es un producto con todas las ``flechas'' en
sentido opuesto.
\begin{definicion}
  Sea \(C , X , Y \in \mathbf O\) objetos arbitrarios de un categoria
  tal que existen \(\phi_x : X \to C,\ \phi_y : Y \to C\) morfismos en
  \(\mathbf M\). Si para todo objeto \(M \in \mathbf O\) tal que posea
  morfimos \( f : X \to C,\ g : Y \to C\) existe un morfismo \(\theta :
  C \to M\) tal que
  \[ f = \theta \circ \phi_x \quad g = \theta \circ \phi_y \]
  Entonces \(C\) es un co-producto de \(X\) e \(Y\).
\end{definicion}
% TODO: considerar incluir diagramas conmutativos
Intuitivamente, pensamos que el producto categorico en
\(\mathscr{HoTop}_*\) deberia sea el producto cartesiano de espacios.
Esto es correcto notanto que \(\forall (X,x_0) , (Y, y_0) \in \mathbf O
\), el espacio \((X \times Y, (x_0, y_0))\) posee los morfismos
(continuos) \(\pi_x : X \times Y \to X \) e \(\pi_y : X \times Y \to Y
\) proyecciones por coordenadas, ademas de una forma de completar el
diagrama conmutativo.

Pero existe otra operacion no tan conocida que puede cumplir el papel
del co-producto solo en espacio puntuados, conocida como la \emph{wedge-sum}
\begin{definicion}
  Sean \((X, x_0),(Y, y_0)\) does espacios topologicos puntuados. El
  espacio cuociente de la union disjunta de \(X\) e \(Y\) con la
  identificacion de punto base \(x_0 \sim y_0\)
  \[ X \vee Y = (X \sqcup Y) / \sim \]
  es conocida como la \emph{Wedge sum} entre \(X\) e \(Y\).
\end{definicion}
La union disjunta de una familia \(\{X_i\}_{i \in \Lambda}\) corresponde
al conjunto
\[ \bigsqcup_{i \in \Lambda} X_i = \{ (x, i) \mid x \in X_i \} \]
es decir, una union donde se marca el conjunto de procedencia en los
elementos. Con esto, es facil ver que dados \(X, Y\) espacios topologicos
puntuados, los morfismos
\[ \phi_x (x) = [(x,0)] \quad \phi_y (y) = [(y,1)] \]
son las inyecciones del co-producto y cumplen tambien las propiedades
universales.

La presencia de estas operaciones, vuelven a la categoria
\(\mathscr{HoTop}_*\) cartesiana y co-cartesiana. Mas aun, podemos
construir equivalencias homotopicas para \(X \times Y,\ X \vee Y\)
siempre y cuando conozcamos equivalencias para \(X\) e \(Y\) mediante
las propiedades universales del (co)-producto.

\begin{teorema}
  Propiedades de producto y co-producto
  % tuve que poner esta linea porque no podia utilizar newline
  \begin{itemize}
    \item Sean \(A, X, Y\) espacio topologicos. Si \(f_0 \sim f_1 : A \to
      X\) y \(g_0 \sim g_1 : A \to Y\), entonces \((f_0 , g_0) \sim (f_1
      , g_1): A \to X \times Y\)
    \item Sean \(B, X, Y\) espacio topologicos. Si \(f_0 \sim f_1 : X \to
      B\) y \(g_0 \sim g_1 : Y \to B\), entonces \(\{f_0 , g_0\} \sim
      \{f_1 , g_1\} : X \vee Y \to B\)
  \end{itemize}
\end{teorema}
\begin{proof}
Comenzando por el producto, sea \(F : A \times I \to X\) la homotopia
entre \(f_0 , f_1\) y sea \(G : A \times I \to Y\)
correspondientemente con \(g_0, g_1\), entonces la homotopia
\begin{align*}
  FG : A \times I &\to X \times Y \\
       (a, t) &\mapsto (F(x,t) , G(x,t))
\end{align*}
Es una homotopia entre \((f_0 , g_0)\) y \((f_1 , g_1)\).

Analogamente para el co-producto. Sea \(F : X \times I \to B\) y \(G : Y
\times I \to B\) las homotopias correspondientes a \(f_0 , f_1\) y \(g_0
, g_1\). Podemos construir la homotopia
\begin{align*}
  FG : X \vee Y \times I &\to B \\
  \big((a,i) , t \big) &\mapsto
                \begin{cases}
                  F(a,t) & i = 0 \\
                  G(a,t) & i = 1 \\
                \end{cases}
\end{align*}
obteniendo asi el resultado deseado.
\end{proof}

\paragraph{Toro.} Dado que en topologia este espacio es definido como
\(S^1 \times S^1\), el teorema anterior nos dice que su grupo
fundamental debe de corresponder a \(\mathbb Z \times \mathbb Z\),
independiente del punto de origen por ser arco-conexo.

Pero aparte de este sencillo ejemplo, nos gustaria conocer teoremas mas
generales que nos respeten la estructura del (co)-producto para
simplificar calculos de grupos fundamentales, esto sera el enfoque de la
siguiente seccion.