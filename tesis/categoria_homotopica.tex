\section{Categoria de homotopias}
En la seccion anterior, se termino dando una definicion de categoria con
\(\mathscr{Top}_*\) y el functor \(_{{*}} : \mathscr{Top} \to
\mathscr{Grp}\). Los morfismos en \(\mathscr{Top}_*\) correspondian a
homeomorfismos entre espacios topologicos, los cuales inducian bajo
\(_{(*)}\) isomorfismos de grupos, pero hay espacios que a pesar de no
ser homeomorfos entre ellos poseen el mismo grupo fundamental y para
estos el functor entre \(\mathscr{Top}_* \to \mathscr{Grp}\) es ciego a
su estructura. Esto es insatisfactorio pensando en la meta
original de clasificar diferentes espacio topologicos y nos hace pensar
en que talvez la nocion de homeomorfismo es mucho pedir entre los
espacios.

Entre las nociones mas populares entre aquellas mas debiles que un
homeomorfismo se encuentra las \emph{equivalencias homotopicas}. Su
popularidad se debe a que corresponden a los morfismos en una nueva
estructura categorica denotada por \(\mathscr{HoTop}_*\) y que contiene
como sub-categoria a \(Top_*\), es decir todo homeomorfismo da espacio a
una equivalencia homotopica. La motivacion de su construccion es natural
a partir del estudio de retracciones, las cuales estudiaremos y iremos
generalizando debidamente.

\subsection{Retracciones}
Iniciaremos estudiando un pequeño lema tecnico con respecto a la
composicion y la funcion identidad.
\begin{lema} \label{thm:comp-identidad}
  Sea \(f : X \to Y\) e \(g : Y \to X\) dos funciones continuas. Si \( g
  \circ f = Id : X \to X \), entonces \(f\) es inyectiva y \(g\) es
  sobreyectiva.
\end{lema}
% TODO: tal vez deberia de probarlo para todo \(\alpha\) biyeccion en
% vez de solo la identidad.
\begin{proof}
  Se argumenta por contradiccion. Supongamos que \(f\) no es inyectiva o
  que \(g\) no es sobreyectiva. Tomando el primer caso para \(f\) se
  tiene que existen \(x_1 , x_2 \in X,\ x_1 \neq x_2\) que cumplen
  \[ f (x_1) = f(x_2) \]
  Dado que \(g\) es una funcion, debe de cumplirse
  \[ g (f (x_1)) = g (f(x_2)) \]
  \[ x_1 = x_2 \]
  Lo cual es una contradiccion.

  Por otro lado, suponiendo que \(g\) no es sobreyectiva, deberia de
  existir \(x \in X\) tal que no exista \( y \in Y\) que \(g (y) = x\).
  Pero por hipotesis, el elemento \(f(x) \in Y\) es tal que \(g (f (x))
  = x\). Mostrando asi que \(g\) es en efecto sobreyectiva.
\end{proof}
Con esto en mente, podemos pasar directamente a la definicion de retracion.
\begin{definicion}
  Sea \(X\) un espacio topologico. \(A \subset X\) es una retraccion de
  \(X\) si existe un mapeo continuo \(r : X \to A\) tal que
  \[ r \mid_{A} (x) = x \]
  En tal caso \(r\) es llamada la aplicacion retraccion de \(X\) en \(A\).
\end{definicion}
Adicionalmente podemos definir trivialmente una inclusion \(j : A \to
X\). Con estas functiones tenemos el siguiente teorema
\begin{teorema}
Si \(A \subset X\) es una retraccion, entonces la inclusion \(j : A \to
X\) induce un homomorfismo de grupos fundamentales \(j_{*} : \pi(A, a)
\to \pi(X,a)\) inyectivo.
\end{teorema}
\begin{proof}
  La composicion \(r \circ j : A \to A\) es la funcion identidad de
  \(A\), por el lema \ref{thm:comp-identidad}, sabemos que \(j\) debe de
  ser inyectivo. Su homorfismo inducido
  \[ (r \circ j)_{*} = r_{*} \circ j_{*} \]
  es el homorfismo identidad entre \(\pi(A,a) \to \pi(A,a)\). Otra vez
  por el lema \ref{thm:comp-identidad}, esto implica que \(r_{*}\) es una
  sobreyeccion y que \(j_{*}\) es inyectiva.
\end{proof}
La retraccion entonces nos da un embedimiento del grupo fundamental de
\( A \) en \(X\). Ya podemos ver un poco de la consecuencias de este
resultado, por ejemplo diciendonos gracias a su contrapositivo que no
existe una retracion de \(B^2\) en \(S^1\), ya que si lo hubiera, la
inclusion \(j : S^1 \to B^2\) seria inyectiva, pero el grupo fundamental
de \(B^2\) es trivial y el de \(S^1\) no lo es (corresponde a \((\mathbb
Z, (+))\)). Esto ultimo es un resultado que veremos mas adelante mediante
espacios cubrimientos.

Entre retracciones que son conocidas, estan \(\mathbb R ^2 - \{0\}\) en
\(S^1\) mediante la funcion \(r (x) = x / \lVert x \rVert \). Lo que nos
dice por el resultado anterior que \(j_{*} : \pi (S^1, a) \to \pi
(\mathbb R ^2 - \{0\}, a)\) es inyectiva. Si mediante algun resultado
pudieramos probar que \(j_{*}\) es ademas sobreyectiva tendriamos listo
nuestro isomorfismo de grupo. Esto es posible de construir pero para eso
necesitamos algunos resultados tecnicos.

\begin{lema} \label{lem:homotopic-inducing}
  Sean \(h,k : (X, x_0) \to (Y, y_0)\) dos mapeos continuous. Si \(h\) y
  \(k\) son homotopicos y si la imagen de \(x_0\) bajo este permanece
  fija en \(y_0\) durante la homotopia, entonces \(h_*\) e \(k_*\) son iguales.
\end{lema}
% TODO: hay que ser mas preciso porque debemos mantener la homotopia
% fija en un punto. Hay que hacer enfasis en el grupo fundamental
% basandose en lo fijo de dicho punto.
\begin{proof}
  Queremos mostrar que \(\forall [f] \in \pi (X,x_0)\), se cumple que
  \(h_* ([f]) = k_* ([f])\). Esto equivale a mostrar que
  \[ [h \circ f] = [k \circ f] \]
  Es decir, debemos de encontrar una homotopia entre \((h \circ f)\) e \(
  (k \circ f)\).

  Para esto, usamos la homotopia \(H : X \times I \to Y \) entre \(h\) y
  \(k\). Notando que \(f : I \to X\) podemos construir una homotopia \(M
  : I \times I \to Y \) pre-componiendo como
  \[ (z, t) \mapsto H \circ (f(z), t) \]
  La cual es una homotopia entre \((h \circ f)\) e \((k \circ f)\) y que
  cumple \( \forall t \in I,\ M (0, t) = y_0\). Por tanto \(h_* = k_*\).
\end{proof}
\begin{teorema} \label{thm:comp-identidad-homotopia}
  Sea \(f : X \to Y\) y \(g : Y \to X\) funciones continuas. Si \(f
  \circ g\) es homotopico a la identidad \( Id : Y \to Y\) y existe
  \(y_0 \in Y\) tal que \( f \circ g (y_0) = y_0 \) entonces
  \(f_*\) es sobreyectivo y \(g_*\) es inyectivo.
\end{teorema}
\begin{proof}
  Aplicando el lema \ref{lem:homotopic-inducing} sobre \(f \circ g \) y
  \( Id : Y \to Y\), se obtiene la igualdad
  \[ (f \circ g)_{*} = f_* \circ g_* = Id_* \]
  Donde \(Id_* : \pi (Y, y_0) \to \pi (Y, y_0)\). Luego aplicando el lema
  \ref{thm:comp-identidad} obtenemos que \(f_*\) es sobreyectiva y que
  \(g_*\) es inyectiva.
\end{proof}
% \begin{proof}
%   Se procede de manera analoga al teorema \ref{thm:comp-identidad}.
%   Supongamos que \(g_*\) no es inyectiva, es decir existen \([a],[b] \in
%   \pi (Y,y_0), [a] \neq [b]\) tales que
%   \[ g_* ([a]) = g_* ([b]) \]
%   Dado que \(f_*\) es una funcion y por tanto solo depende sus
%   argumentos, aplicando a ambos lados \(f_*\)
%   \[ f_* (g_* ([a])) = f_* (g_* ([b])) \]
%   \[ [a] = [b] \]
%   Lo cual es una contradiccion con la suposicion inicial.

%   De manera analoga, para mostrar que \(f_*\) es sobreyectiva,
%   supongamos que existe \([c] \in \pi (Y, y_0)\) tal que no existe \([d]
%   \in \pi (X,x_0)\) tal que \(g_* ([d]) = [c]\). Pero por cumplirse que
%   \( f_* \circ g_* = Id_* : \pi (Y, y_0) \to \pi (Y, y_0)\), sabemos que
%   existe \(g_* ([c])\) que cumple dicha relacion, por tanto \(f_*\) es
%   sobreyectiva.
% \end{proof}
Esta es exactamente la misma demostracion que en el teorema
\ref{thm:comp-identidad}, pero haciendo enfasis lo que significa ser
inyectivo o sobreyectivo en clases de equivalencias. Resulta que ya
tenemos todos los resultados necesarios para expandir el resultado de la
retraccion \(\mathbb R ^2 - \{0\}\) en \(S^1\).

\begin{teorema}
  La inclusion homomorfica \(j_* : \pi (S^1, a) \to \pi (\mathbb R ^2 -
  \{0\}, a)\) es un isomorfismo de grupos fundamentales.
\end{teorema}
\begin{proof}
  Ya habiamos dicho anteriormente que \( j_* : \pi (S^1, a) \to \pi
  (\mathbb R ^2 - \{0\}, a)\) es inyectiva como homomorfismo de grupos
  fundamentales, puesto la existencia de la retraccion
  \begin{align*}
    r : \mathbb R ^2 - \{0\} &\to S^1 \\
    x &\mapsto \frac x {\lVert x \rVert}
  \end{align*}
  implica por teorema \ref{thm:comp-identidad} la inyectividad.

  Para probar la sobreyectividad de \(j_*\), tomemos la composicion
  \[ j \circ r : \mathbb R ^2 - \{0\} \to \mathbb R ^2 - \{0\} \]
  Esta composicion \emph{no} es la identidad de \(\mathbb R ^2 - \{0\}
  \) pero es homotopica a esta mediante la homotopia de linea recta
  \[ H(x,t) := t \cdot x + (1 - t) \cdot \frac x {\lVert x \rVert} \]
  Esta homotopia deja fijo al punto \((1,0)\), luego tenemos las
  hipotesis del teorema \ref{thm:comp-identidad-homotopia} y obtenemos
  que \(j_*\) es ademas sobreyectiva. Concluimos que \(j_*\) es un
  isomorfismo de grupos fundamentales.
\end{proof}

\subsection{Tipos homotopicos}
Ahora que ya conocemos el esquema de trabajo, podemos generalizar para
espacios que no necesariamente sean retracciones entre si, pero que
posean un par de funciones cuyas composiciones sean homotopicas a la
identidad correspondiente.
\begin{definicion}
  Sean \(f : X \to Y\) e \(g : Y \to X\) mapeos continuos. Supongamos
  que \( g \circ f : X \to X \) es homotopico al mapeo identidad de
  \(X\) y \( f \circ g : Y \to Y \) es homotopico al mapeo identidad de
  \(Y\). Entonces \(f\) y \(g\) son llamadas \emph{equivalencias
  homotopicas} y cada una es una \emph{inversa homotopica} de la otra.
  Si dos espacio poseen un par de equivalencias homotopicas entre ellos,
  se dicen que ambos espacios son del mismo \emph{tipo homotopico}.
\end{definicion}
Podemos ver que en el caso anterior, la retraccion y la inclusion
correspondian a par de equivalencias homotopicas. Podria pensarse que la
existencia de un par de equivalencias homotopicas entre espacios seria
suficiente para afirmar que poseen el mismo grupo fundamental, esto es
cierto en medida que se sea cuidadoso con los puntos de partida. Este
problema no se manifestaba en retracciones porque el punto de partida se
mantenia fijo en la inclusion. Para ver como afecta, seremos mas
especificos con la notacion de levantamiento : Un levantamiento de \(f :
X \to Y\) con respecto a \(x_0\) corresponde a la funcion
\[ (f_{x_0}) : \pi (X , x_0) \to \pi (Y, y_0)\]
Donde \(f(x_0) = y_0\). Ahora, supongamos que tenemos dos espacio \(X,
Y\) junto con una equivalencias homotopicas
\[ f : X \to Y \quad g : Y \to X \]
Por hipotesis tenemos que \(g \circ f \simeq Id : X \to X\), pero eso
\textbf{no} nos asegura la existencia de un punto \(x_0 \in X\) que sea
fijo bajo la homotopia entre \( g \circ f\) y \(Id\). Por tanto no
podemos aplicar el lema \ref{lem:homotopic-inducing} pues no podemos
cumplir una de las hipotesis. Resulta que esto es subsanable utilizando
el siguiente lema tecnico.

\begin{lema} \label{lem:equiv-hom-lift}
  Sean \(h,k : X \to Y\) mapeos continuos, sea \(h (x_0) = y_0,\ k(x_0)
  = y_1\). Si \(h\) y \(k\) son homotopicos mediante \(H : X \times I
  \to Y\), entonces existe un arco \(\alpha\) en \(Y\) de \(y_0\) a
  \(y_1\) tal que \(k_* = \hat \alpha \circ h_* \). Mas aun, este arco
  esta dado por la ecuacion \(\alpha (t) = H (x_0, t)\).
\end{lema}
\noindent Aqui se utilizara la notacion \(\hat \alpha\) que se utilizo
pero no definicion en la demostracion de el teorema \ref{not:alpha-hat}.
\begin{proof}
  Hemos de probar que \(\forall [f] \in \pi (X, x_0)\) se cumple que
  \[ k_* ([f]) = \hat \alpha \circ h_* ([f]) \]
  \[ [k \circ f] = [ \alpha^{-1} ] *  [h \circ f] * [\alpha] \]
  \[ [ \alpha ] * [k \circ f] =  [h \circ f] * [\alpha] \]
  Es decir, hemos de construir una homotopia entre estos dos
  representantes de clase. Sea \(H : X \times I \to Y\) la homotopia
  entre \(h\) y \(k\) dada por hipotesis, entonces podemos reescribir
  las funciones de la ultima ecuacion definiendo nuevas curvas sobre \(I
  \times I\)
  \[ f_0(t) := (f(t), 0) \quad H \circ f_0 = h \circ f \]
  \[ f_1(t) := (f(t), 1) \quad H \circ f_1 = k \circ f \]
  \[ c(t) := (x_0, t) \quad H \circ c = \alpha \]

  De igual manera re-definiremos las tres nuevas curvas en terminos de
  \(F : I \times I \to X \times I\) dada por la ecuacion \(F(s,t) =
  (f(s),t)\). Para esto debemos considerar los cuatro bordes de
  \(I \times I\).
  \[ \beta_0(s) = (s, 0) \quad \beta_1(s) = (s, 1)\]
  \[ \gamma_0(t) = (0, t) \quad \gamma_1(t) = (1, t)\]
  Mostrandose las siguientes igualdades
  \[ F \circ \beta_0 = f_0 \quad F \circ \beta_1 = f_1\]
  \[ F \circ \gamma_0 = c = F \circ \gamma_1 \]

  Notese que los caminos \(\beta_0 * \gamma_1\) y \(\gamma_0 * \beta_1\)
  son los caminos inferior derecho y izquierdo superior del cuadrilatero
  \(I \times I\). Dada la convexidad de este espacio, existe una
  homotopia denotada por \(G : I \times I \to I \times I \). Luego por
  composicion \(F \circ G\) es una homotopia entre \(f_0 * c\) y \(c *
  f_1\). Para finalizar vemos que la composicion \(H \circ F \circ G\)
  es una homotopia en \(Y\)
  \[ (H \circ f_0) * (H \circ c) = (h \circ f) * \alpha \]
  \[ (H \circ c) * (H \circ f_1) = \alpha * (k \circ f) \]
  Obteniendo asi la homotopia buscada.
\end{proof}

La importancia de encontrar dicho arco \(\alpha : I \to Y\) entre los
dos levantamientos proviene del siguiente lema.
\begin{lema} \label{lem:hat-alpha-iso}
\(\hat \alpha : \pi (Y, y_0) \to \pi (Y, y_1)\) es un isomorfismo de
grupos fundamentales.
\end{lema}
\begin{proof}
  Por definicion \(\hat \alpha\) es claramente un homomorfismo. Para ver
  que es inyectiva, supongamos que tomamos \([f],[g] \pi (Y, y_0),\ [f]
  \neq [g]\) tal que se cumple \(\hat \alpha ([f]) = \hat \alpha
  ([g])\). Esto implicaria por reduccion que
  \[ [\alpha^{-1}] * [f] * [\alpha] = [\alpha^{-1}] * [g] * [\alpha] \]
  \[ [f] * [\alpha] = [g] * [\alpha] \]
  \[ [f] = [g] \]
  Lo cual es una contradiccion

  Para ver la sobreyectividad, para todo elemento \([h]
  \in \pi (Y,y_0)\) existe \( [\alpha] * [h] * [\alpha^{-1}] \in \pi
  (Y,y_0)\) tal que
  \[ \hat \alpha ([\alpha] * [h] * [\alpha^{-1}]) = [h]\]
\end{proof}
Con esto es mente, es facil ver que se tiene el siguiente corolario
\begin{corolario}
  Sean \(h,k : X \to Y\) homotopicamente continuas tales que \(h (x_0) =
  y_0\) y \(k(x_0) = y_1\). Si \(h_*\) es inyectiva o sobreyectiva o
  trivial, entonces \(k_*\) tambien lo es.
\end{corolario}
\begin{proof}
  Dado que por el lema \ref{lem:equiv-hom-lift} tenemos la siguiente
  ecuacion entre levantamientos
  \[ k_* = \hat \alpha \circ h_* \]
  Por el lema \ref{lem:hat-alpha-iso} sabemos que \(\hat \alpha\) es
  biyectivo. Por tanto la composicion de \(h_*\) inyectiva con \(\hat
  \alpha\) bijectiva es inyectiva. De igual manera para sobreyectividad y
  la trivialidad.
\end{proof}
Con esto podemos al fin plantear que todo par de espacios en el mismo
tipo homotopico poseen el mismo grupo fundamental. Esto gracias a que
podemos evitar la condicion de punto fijo en el lema
\ref{lem:homotopic-inducing} gracias a la biyeccion de \(\hat \alpha\).

\begin{teorema}
  Sean \(f : X \to Y\) continua y \(f (x_0) = y_0\). Si \(f\) es una
  homotopia equivalencia con \(g : Y \to X\), entonces
  \[ f_* : \pi (X, x_0) \to \pi (Y, y_0)\]
  es un isomorfismo.
\end{teorema}
\begin{proof}
  Fijemos los puntos \(x_1 = g(y_0)\) y \(y_1 = f(x_1)\). Vease que dado
  los diferentes puntos bases se tienen dos homomorfismos inducidos de
  \(f\) con distintos puntos relativos \[f_{x_0} : \pi (X, x_0) \to \pi
  (Y, y_0)\] \[f_{x_1} : \pi (X, x_1) \to \pi (Y, y_1)\].

  Por hipotesis, \(g \circ f : (X, x_0) \to (X, x_1)\) es homotopico a
  la funcion identidad \(Id : X \to X\). Luego por el lema
  \ref{lem:equiv-hom-lift} se tiene que
  \begin{equation} \label{eq:alpha-f-g}
  g_* \circ (f_{x_0})_* = (g \circ f)_* = \hat \alpha \circ (Id_X)_*
    = \hat \alpha
  \end{equation}
  Por tanto \(g_* \circ (f_{x_0})_*\) es un isomorfismo y esto obliga a
  que \(g_*\) sea sobreyectivo.

  De igual manera, aplicando el analisis anterior a \(f \circ g : \pi
  (Y, y_0) \to (Y, y_1)\) homotopico a la identidad de \(Y\), se
  obtiene que \((f_{x_1})_* \circ g_*\) es un isomorfismo y por tanto
  \(g_*\) es inyectivo tambien.

  Conociendo la biyectividad de \(g_*\), nos permite pre-componer por su
  inversa \((g_*)^{-1}\) en la ecuacion \eqref{eq:alpha-f-g}, obteniendo
  \[ (f_{x_0})_* = (g_*)^{-1} \circ \hat \alpha\]
  luego, la composicion de biyeccion es biyeccion y por tanto
  \((f_{x_0})_*\) es un isomorfismo de grupos fundamentales.
\end{proof}

\subsection{La categoria \(\mathscr{HoTop}_*\)}
\begin{definicion}
  La categoria \(\mathscr{HoTop}_*\) corresponde a una tripleta
  \((\mathbf{O},\mathbf M, \circ)\) donde
  \begin{itemize}
  \item \(\mathbf O\) corresponde a espacio topologicos puntuados de la
    forma \((X, x_0)\).
  \item \(\mathbf M\) corresponde a un conjunto
    \[ M = \{ [f] \mid f : (X,x_0) \to (Y,y_0),\ (X,x_0),(Y,y_0) \in
      \mathbf O \}\]
    de clases equivalencias homotopicas entre funciones continuas
  \item \((\circ)\) corresponde a la composicion de clases de
    equivalencias tal que si \([f] , [g] \in \mathbf O \) tal que
    \[ f : (X,x_0) \to (Y, y_0)\]
    \[ g : (Y, y_0) \to (Z, z_0)\]
    entonces \([g] \circ [f]\) es una clase de equivalencia de \((X,
    x_0) \to (Z, z_0)\)
  \end{itemize}

\end{definicion}