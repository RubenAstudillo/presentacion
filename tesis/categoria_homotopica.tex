\section*{Categoria de homotopias}
En la seccion anterior, se termino dando una definicion de categoria con
\(\mathscr{Top}_*\) y el functor \(_{{*}} : \mathscr{Top} \to
\mathscr{Grp}\). Los morfismos en \(\mathscr{Top}_*\) correspondian a
homeomorfismos entre espacios topologicos, los cuales inducian bajo
\(_{(*)}\) isomorfismos de grupos, pero hay espacios que a pesar de no
ser homeomorfos entre ellos poseen el mismo grupo fundamental y para
estos el functor entre \(\mathscr{Top}_* \to \mathscr{Grp}\) es ciego a
su estructura. Esto es insatisfactorio pensando en la meta
original de clasificar diferentes espacio topologicos y nos hace pensar
en que talvez la nocion de homeomorfismo es mucho pedir entre los
espacios.

Entre las nociones mas populares entre aquellas mas debiles que un
homeomorfismo se encuentra las \emph{equivalencias homotopicas}. Su
popularidad se debe a que corresponden a los morfismos en una nueva
estructura categorica denotada por \(\mathscr{HoTop}_*\) y que contiene
como sub-categoria a \(Top_*\), es decir todo homeomorfismo da espacio a
una equivalencia homotopica. La motivacion de su construccion es natural
a partir del estudio de retracciones, las cuales estudiaremos y iremos
generalizando debidamente.

\subsection*{Retracciones}
Iniciaremos estudiando un pequeño lema tecnico con respecto a la
composicion y la funcion identidad.
\begin{lema} \label{thm:comp-identidad}
  Sea \(f : X \to Y\) e \(g : Y \to X\) dos funciones continuas. Si \( g
  \circ f = Id : X \to X \), entonces \(f\) es inyectiva y \(g\) es
  sobreyectiva.
\end{lema}
\begin{proof}
  Se argumenta por contradiccion. Supongamos que \(f\) no es inyectiva o
  que \(g\) no es sobreyectiva. Tomando el primer caso para \(f\) se
  tiene que existen \(x_1 , x_2 \in X,\ x_1 \neq x_2\) que cumplen
  \[ f (x_1) = f(x_2) \]
  Dado que \(g\) es una funcion, debe de cumplirse
  \[ g (f (x_1)) = g (f(x_2)) \]
  \[ x_1 = x_2 \]
  Lo cual es una contradiccion.

  Por otro lado, suponiendo que \(g\) no es sobreyectiva, deberia de
  existir \(x \in X\) tal que no exista \( y \in Y\) que \(g (y) = x\).
  Pero por hipotesis, el elemento \(f(x) \in Y\) es tal que \(g (f (x))
  = x\). Mostrando asi que \(g\) es en efecto sobreyectiva.
\end{proof}
Con esto en mente, podemos pasar directamente a la definicion de retracion.
\begin{definicion}
  Sea \(X\) un espacio topologico. \(A \subset X\) es una retraccion de
  \(X\) si existe un mapeo continuo \(r : X \to A\) tal que
  \[ r \mid_{A} (x) = x \]
  En tal caso \(r\) es llamada la aplicacion retraccion de \(X\) en \(A\).
\end{definicion}
Adicionalmente podemos definir trivialmente una inclusion \(j : A \to
X\). Con estas functiones tenemos el siguiente teorema
\begin{teorema}
Si \(A \subset X\) es una retraccion, entonces la inclusion \(j : A \to
X\) induce un homomorfismo de grupos fundamentales \(j_{*} : \pi(A, a)
\to \pi(X,a)\) inyectivo.
\end{teorema}
\begin{proof}
  La composicion \(r \circ j : A \to A\) es la funcion identidad de
  \(A\), por el lema \ref{thm:comp-identidad}, sabemos que \(j\) debe de
  ser inyectivo. Su homorfismo inducido
  \[ (r \circ j)_{*} = r_{*} \circ j_{*} \]
  es el homorfismo identidad entre \(\pi(A,a) \to \pi(A,a)\). Otra vez
  por el lema \ref{thm:comp-identidad}, esto implica que \(r_{*}\) es una
  sobreyeccion y que \(j_{*}\) es inyectiva.
\end{proof}
La retraccion entonces nos da un embedimiento del grupo fundamental de
\( A \) en \(X\). Ya podemos ver un poco de la consecuencias de este
resultado, por ejemplo diciendonos gracias a su contrapositivo que no
existe una retracion de \(B^2\) en \(S^1\), ya que si lo hubiera, la
inclusion \(j : S^1 \to B^2\) seria inyectiva, pero el grupo fundamental
de \(B^2\) es trivial y el de \(S^1\) no lo es (corresponde a \((\mathbb
Z, (+))\)). Esto ultimo es un resultado que veremos mas adelante mediante
espacios cubrimientos.

Entre retracciones que son conocidas, estan \(\mathbb R ^2 - \{0\}\) en
\(S^1\) mediante la funcion \(r (x) = x / \lVert x \rVert \). Lo que nos
dice por el resultado anterior que \(j_{*} : \pi (S^1, a) \to \pi
(\mathbb R ^2 - \{0\}, a)\) es inyectiva. Si mediante algun resultado
pudieramos probar que \(j_{*}\) es ademas sobreyectiva tendriamos listo
nuestro isomorfismo de grupo. Esto es posible de construir pero para eso
necesitamos algunos resultados tecnicos.

\begin{lema}
  Sean \(h,k : (X, x_0) \to (Y, y_0)\) dos mapeos continuous. Si \(h\) y
  \(k\) son homotopicos y si la imgen de \(x_0\) bajo este permanece
  fija en \(y_0\) durante la homotopia, entonces \(h_*\) e \(k_*\) son iguales.
\end{lema}
\begin{proof}
  Queremos mostrar que \(\forall [f] \in \pi (X,x_0)\), se cumple que
  \(h_* ([f]) = k_* ([f])\). Esto equivale a mostrar que
  \[ [h \circ f] = [k \circ f] \]
  Es decir, debemos de encontrar una homotopia entre \((h \circ f)\) e \(
  (k \circ f)\).

  Para esto, usamos la homotopia \(H : X \times I \to Y \) entre \(h\) y
  \(k\). Notando que \(f : I \to X\) podemos construir una homotopia \(M
  : I \times I \to Y \) pre-componiendo como
  \[ (z, t) \mapsto H \circ (f(z), t) \]
  La cual es una homotopia entre \((h \circ f)\) e \((k \circ f)\) y que
  cumple \( \forall t \in I,\ M (0, t) = y_0\). Por tanto \(h_* = k_*\).
\end{proof}
\begin{lema}
  Sea \(f : X \to Y\) y \(g : Y \to X\) funciones continuas. Si \(f_*
  \circ g_*\) es homotopico a la identidad \( Id : Y \to Y\), entonces
  \(f_*\) es sobreyectivo e \(g_*\) es inyectivo.
\end{lema}
\begin{proof}
  Se procede de manera analoga al teorema \ref{thm:comp-identidad}.
  Supongamos que \(g_*\) no es inyectiva, es decir existen \([a],[b] \in
  \pi (Y,y_0), [a] \neq [b]\) tales que
  \[ g_* ([a]) = g_* ([b]) \]
  Dado que \(f_*\) es una funcion y por tanto solo depende sus
  argumentos, aplicando a ambos lados \(f_*\)
  \[ f_* (g_* ([a])) = f_* (g_* ([b])) \]
  \[ [a] = [b] \]
  Lo cual es una contradiccion con la suposicion inicial.

  De manera analoga, para mostrar que \(f_*\) es sobreyectiva,
  supongamos que existe \([c] \in \pi (Y, y_0)\) tal que no existe \([d]
  \in \pi (X,x_0)\) tal que \(g_* ([d]) = [c]\). Pero por cumplirse que
  \( f_* \circ g_* = Id_* : \pi (Y, y_0) \to \pi (Y, y_0)\), sabemos que
  existe \(g_* ([c])\) que cumple dicha relacion, por tanto \(f_*\) es
  sobreyectiva.
\end{proof}
Esta es exactamente la misma demostracion que en el teorema
\ref{thm:comp-identidad}, pero haciendo enfasis lo que significa ser
inyectivo o sobreyectivo en clases de equivalencias. Resulta que ya
tenemos todos los resultados necesarios para expandir el resultado de la
retraccion \(\mathbb R ^2 - \{0\}\) en \(S^1\).

\begin{teorema}
  El inclusion homomorfica \(j_* : \pi (S^1, a) \to \pi (\mathbb R ^2 -
  \{0\})\) es un isomorfismo de grupos fundamentales.
\end{teorema}