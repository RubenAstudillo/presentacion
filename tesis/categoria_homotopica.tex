\section*{Categoria de homotopias}
En la seccion anterior, se termino dando una definicion de categoria con
\(\mathscr{Top}_*\) y el functor \(_{{*}} : \mathscr{Top} \to
\mathscr{Grp}\). Los morfismos en \(\mathscr{Top}_*\) correspondian a
homeomorfismos entre espacios topologicos, los cuales inducian bajo
\(_{(*)}\) isomorfismos de grupos, pero hay espacios que a pesar de no
ser homeomorfos entre ellos poseen el mismo grupo fundamental y para
estos el functor entre \(\mathscr{Top}_* \to \mathscr{Grp}\) es ciego a
su estructura. Esto es insatsifactorio desde pensando en la meta
original de clasificar diferentes espacio topologicos y nos hace pensar
en que talvez la nocion de homeomorfismo es mucho pedir entre los
espacios.

Entre las nociones mas populares entre aquellas mas debiles que un
homeomorfismo se encuentra las \emph{equivalencias homotopicas}. Su
popularidad se debe a que corresponden a los morfismos en una nueva
estructura categorica denotada por \(\mathscr{HoTop}_*\) y que contiene
como sub-categoria a \(Top_*\), es decir todo homeomorfismo da espacio a
una equivalencia homotopica. La motivacion de su construccion es natural
a partir del estudio de retracciones, las cuales estudiaremos y iremos
generalizando debidamente.

\subsection*{Retracciones}
Las retracciones poseen buenas propiedades para establecer inclusion de
grupos fundamentales
\begin{definicion}
  Sea \(X\) un espacio topologico. \(A \subset X\) es una retraccion de
  \(X\) si existe un mapeo continuo \(r : X \to A\) tal que
  \[ r \mid_{A} (x) = (x) \]
  En tal caso \(r\) es llamada la aplicacion retraccion de \(X\) en \(A\).
\end{definicion}