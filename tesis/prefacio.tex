\section{Prefacio}
Topología nos da un lenguaje para entender propiedades de diferentes
espacios a través de nociones como compacidad, cerrado/abierto y
diferentes tipos
de separabilidad. Gran parte de estas propiedades tienen un carácter
local, es decir no son propiedades que definan al espacio total si no mas
bien por subparte. En el siglo XIX el mundo del álgebra, en
particular la teoría de grupos había intentado y logrado clasificar
progresivamente grupos de diferentes ordenes modulo isomorfismo. Habían
logrado esto a través de llamadas invariantes algebraicas, las cuales
eran preservadas a través de los isomorfismo de grupos. En topología se
deseaba hacer un tratamiento parecido, por lo cual necesitaban una
invariante topológica que nos permitiera clasificar espacios. La elección
de esta invariante no necesariamente nos da teorías útiles, pues puede
ser muy general y no distinguir en propiedades que se consideran
importantes. Uno puede tomar como ejemplo que si nuestra propiedad es
la compacidad, esta no hace un buen papel en distinguir entre
\(\mathbb{R}^2\) y \(\mathbb{R}^3\).

Si hemos de tomar una propiedad como invariante topológica sobre la cual
construir una teoría, esta no debe ser local. Idealmente nos gustaría
construir sobre alguna propiedad del espacio que fuera característica de
todo el espacio, algo en general notoriamente difícil. Resulta que si nos
reducimos a espacios arco-conexos podemos construir una respuesta a través
de los grupos fundamentales. Esta característica es construida sobre
clases de equivalencias de caminos cerrados quienes puedan deformarse
entre si y permiten reificar información de los ``agujeros'' del espacio.
Resulta que la elección de esta respuesta tiene la consecuencia feliz de
tener estructura de grupo subyacente, por lo que nos permite reutilizar
el trabajo ya realizado en el mundo del álgebra para nuestro beneficio.
Mas aun, la construcción en si da cabida a interpretaciones de
teoría de categorías, por lo que sirve de buena introducción a un
lenguaje mas general.
