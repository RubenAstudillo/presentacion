\section{Prefacio}
Topología nos da un lenguaje para definir precisamente nociones como
compacidad, cerrado/abierto y diferentes tipos de separabilidad. Gran
parte de estas propiedades tienen un carácter local, es decir no son
propiedades que definan al espacio total si no mas bien por subparte de
este. En el siglo XIX el mundo del álgebra, en particular la teoría de
grupos, había intentado y conseguido clasificar progresivamente grupos
de diferentes ordenes modulo isomorfismo. Habían logrado esto a través
de llamadas invariantes algebraicas, las cuales eran preservadas a
través de los isomorfismo de grupos. En topología se deseaba hacer un
tratamiento parecido para clasificar espacios, por lo cual se necesitaba
encontrar una invariante topológica. La elecciones de esta no
necesariamente nos dan teorías útiles, pues puede ser muy general y no
distinguir en propiedades que se consideran importantes. Si hemos de
tomar una propiedad como invariante topológica, sobre la cual construir
una teoría, esta no debe ser local. Idealmente nos gustaría construir
sobre alguna propiedad del espacio que fuera característica de todo el
espacio, algo en general notoriamente difícil. Resulta que si nos
reducimos a espacios arco-conexos podemos construir una respuesta
adecuada a través de los grupos fundamentales. Esta característica es
construida sobre clases de equivalencias de caminos cerrados, quienes
puedan deformarse entre si y permiten reificar información de los
``agujeros'' del espacio. Resulta que la elección de esta respuesta
tiene la consecuencia feliz de tener estructura de grupo subyacente, por
lo que nos permite reutilizar el trabajo ya realizado en el mundo del
álgebra para nuestro beneficio. Mas aun, las técnicas utilizadas en la
definición de este, son suficientemente abstractas como para ser
aplicado a otros dominios, lo cual motivo su estudio en propio interés
bajo el nombre de \emph{teoría de categorías}. Por lo que un
entendimiento de la topología algebraica es una buena introducción,
tanto practica como histórica, al lenguaje de categorías.
