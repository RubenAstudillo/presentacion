\section{Prefacio}
En el estudio de los espacios topológicos, la búsqueda de invariantes
topológicas es de interés pues nos permite clasificar espacios a pesar
de que estos no sean isomorfos. Existen varias invariantes de las cuales
elegir, en particular están distintos propiedades de conexidad,
compacidad y separabilidad. Nosotros estudiaremos una invariante
conocida como el grupo fundamental. Esta se basa en el estudio de arcos
o caminos sobre un espacio topológico y nos permite distinguir espacios
según la cantidad de ``agujeros'' en ellos. Esta es relativamente
popular por dos grandes: El grupo fundamental es un grupo algebraico
formado a partir de un espacio topológico, por lo tanto toda la teoría
de grupos y en particular la de clasificación de ellos puede ser
re-utilizada, la segunda viene de que históricamente el estudio de la
relación de entre espacio topológicos y sus grupos fundamentales dio
inicio al estudio y formalización de la teoría de categorías.

Este trabajo tiene a grandes rasgos dos objetivos. El de mostrar alguna
construcciones categóricas mediante el estudio de la topología
algebraica y el de presentar los teoremas mas importante para el calculo
e identificación de grupos fundamentales de distintos espacios.
