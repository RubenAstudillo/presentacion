\subsection{Grupoide de arcos}
Si tenemos dos arcos continuos \(f,g\) tales que el punto final de \(f\)
sea el punto inicial de \(g\) podemos construir un camino continuo que
recorra \(f\) luego recorriendo \(g\). Esta idea es conocida formalmente
como el producto de dos arcos.

\begin{definicion}[Producto de arcos]
Para dos arcos \(f,g : I \to X\) tales que
\(f(1) = g(0)\), se define el producto \(f * g \)
\[ (f*g) (s) \coloneqq \begin{cases}
    f(2s) & s \in [0,\frac{1}{2}] \\
    g(2s - 1) & s \in [\frac{1}{2} , 1]
  \end{cases}
\]
la cual sigue siendo una funcion continua en virtud del lema del
pegamiento.
\end{definicion}

Esta construccion se puede reutilizar para clases
\emph{arco}-homotopicas \([f],[g]\) que compartan punto final e inicial
respectivamente para definir un producto de clases de equivalencia
\[ [f] * [g] \coloneqq [f * g]\]
El cual esta bien definido pues si \(f \simeq f' ,\ g \simeq
g'\) a traves de las homotopias \(F, G\) respectivamente, entonces
\[
  H(s,t) = \begin{cases}
    F(2s,t) & s \in [0, \frac{1}{2}] \\
    G(2s - 1, t) & s \in [\frac{1}{2} , 1]
  \end{cases}
\]
Es la homotopia que relaciona a los arcos que sean homotopicos a
\(f*g\). Esta ademas es continua en virtud otra vez del lema del pegamiento.

\paragraph{} Con esta operacion binaria, una pregunta natural es si
\((\mathcal C (I , X)/\simeq , (*))\) tiene estructura de
grupoide, esto equivale a cumplir 3 propiedades
\begin{enumerate}
\item \textbf{Asociatividad} Si \([f] * ([g] * [h])\) esta definido entonces
  tambien lo debe estar \(([f] * [g]) * [h]\) y ademas deben coincidir.
\item \textbf{Identidades izquierda y derecha} Para todo \([f]\) con
  \(x_0, x_1\) puntos inicial y final respectivamente, debe de
existir elementos \([k_{x_0}], [k_{x_1}]\) tales que
\[ \begin{matrix}
    [f] * [k_{x_1}] = [f] & & [k_{x_0}] * [f] = [f]
  \end{matrix}
\]
\item \textbf{Inverso} Para todo \([f]\) clase de arcos con \(x_0, x_1\)
  puntos inicial y final respectivamente debe de existir un elemento
  \([f^{-1}]\) que cumpla
\[ \begin{matrix}
    [f] * [f^{-1}] = [k_{x_0}] & & [f^{-1}] * [f] = [k_{x_1}]
  \end{matrix}
\]
\end{enumerate}
Para probar esto necesitamos primero definir a nuestros candidatos de
\(k_{x_0}, k_{x_1}, f^{-1}\) ademas de algunas funciones auxiliares,
iniciando por los mapeos constantes e identidad en \(I \to I\)
\[ \begin{matrix}
     e_0 : & I \to I & e_0(t) \coloneqq 0 \\
     e_1 : & I \to I & e_1(t) \coloneqq 1 \\
     i :   & I \to I & i(t) \coloneqq t \\
     \bar{i} : & I \to I & i(t) \coloneqq 1 - t
   \end{matrix}
   \]
Ademas, para todo arco \(f : I \to X \), el elemento \(f^{-1} : I \to X \) esta
definido (en el espiritu de \eqref{eq:homotopy-simetry}) por
\[ f^{-1} (s) \coloneqq f (1 - s) \]
Por ultimo, para todo \(x \in X \) se define la curva
constante\footnote{\(k\) por \emph{konst}}
\begin{align*}
  k_x : &I \to X \\
        &t \mapsto x
\end{align*}
Con esto podemos afrontar la demostración
\begin{proof}
Se procedera en orden \(2 \to 3 \to 1\) con lemas en medio. Sea \(f : I
\to X\) el representante de la clase \([f]\). sean \(x_0, x_1\) los
puntos inicial y final de \(f\) respectivamente.

\paragraph{(2).} Dados \(i,e_1\) arcos en \(I\) definidos anteriormente,
es claro que \(i * e_1\) es tambien un arco (continuo) en \(I\); mas aun,
ya que \(I = [0,1]\) es convexo, se tiene \(i * e_1 \simeq i\) con la
homotopia de la linea recta entre estos denotada por \(C\).

Tambien conocemos que la composición de funciones continuas es
continua, de lo cual podemos decir que dada una homotopia \(H : I \times
I \to I\) la composicion \( f \circ H : I \times I \to X\) es una homotopia.

Por otro lado, necesitamos conocer como se comporta la composicion con
respecto a nuestro producto \((*)\), para esto tenemos el siguiente lema
\begin{lema}[Distributividad de la composicion sobre producto]
\label{lema:dist-composicion-producto}
\[\forall a,b : I \to I,\ \forall f : I \to X,\ f \circ (a * b) = (f
\circ a) * (f \circ b) \]
\end{lema}
\begin{proof}
  \[ f \circ (a*b) (s) =
    \begin{cases}
      f (a(2s)) & s \in [0,\frac{1}{2}] \\
      f (b(2s - 1)) & s \in [\frac{1}{2} , 1]
    \end{cases}
    = (f \circ a) * (f \circ b) (s)
  \]
\end{proof}

Luego esto nos permite afirmar en especifico que
\[ f \circ i \simeq f \circ (i * e_1) = (f \circ i) * (f \circ e_1) \]
en virtud de la homotopia
\[ f \circ C : I \times I \to X \]
Por tanto
\begin{equation}\label{eq:homequiv2.1}
[f \circ i] = [f \circ (i * e_1)] = [(f \circ i) * (f \circ e_1)]
\end{equation}
Si tomamos este ultimo termino, sabemos que por definicion
\begin{equation}\label{eq:homequiv2.2}
[(f \circ i) * (f \circ e_1)] = [(f \circ i)] * [(f \circ e_1)]
\end{equation}
y es claro tambien que
\[
  \begin{matrix}
    f \circ i = f & f \circ e_1 = k_{x_1} \\
  \end{matrix}
\]
\begin{equation}\label{eq:homequiv2.3}
[(f \circ i)] * [(f \circ e_1)] = [f] * [k_{x_1}]
\end{equation}
juntando
\eqref{eq:homequiv2.1},\eqref{eq:homequiv2.2},\eqref{eq:homequiv2.3} nos
muestra que \([k_{x_1}]\) es la identidad por la derecha. Podemos
hacer un proceso analogo para \([k_{x_0}]\) como identidad por la
izquierda, asi probando (2).

\paragraph{(3).} De manera similar a la pregunta anterior, notar que \(i,
\bar{i}\) definidos anteriormente cumplen
\[ i * \bar{i} \simeq e_0 \]
por la homotopia de linea recta en el espacio \(I\) que es convexo.
Utilizando el lemma \eqref{lema:dist-composicion-producto} tenemos que
\[ [f * f^{-1}] = [f \circ (i * \bar{i})] = [f \circ e_0] = [k_{x_0}] \]
\[ \implies [f] * [f^{-1}] = [k_{x_0}] \]
De manera analoga se prueba la inversa por la izquierda.

\paragraph{(1).}
Por definicion, tenemos que las diferentes asociaciones nos resultan en
diferentes representates de clase
\[
  \begin{matrix}
    (f * g) * h \, (t) =
    \begin{cases}
      f (4t), & t \in [0, \frac 1 4] \\
      g (4t - 1), & t \in [\frac 1 4, \frac 1 2] \\
      h (2t - 1), & t \in [\frac 1 2, 1] \\
    \end{cases} &
    f * (g * h) \, (t) =
    \begin{cases}
      f (2t), & t \in [0, \frac 1 2] \\
      g (4t - 2), & t \in [\frac 1 2, \frac 3 4] \\
      h (4t - 3), & t \in [\frac 3 4, 1] \\
    \end{cases}
  \end{matrix}
\]
Sea \(T : I \to I\) definida por
\[ T (t) :=
  \begin{cases}
    2t & t \in [1 , \frac 1 4] \\
    t + \frac 1 4 & t \in [\frac 1 4, \frac 1 2] \\
    \frac t 2 + \frac 1 2 & t \in [\frac 1 2, 1]
  \end{cases}
\]
\(T\) es una funcion continua por el lema del pegamiento. Por ser \(I\)
un dominio convexo, se tiene que \(i \simeq T\), luego
\[ [(f * g) * h] = [\big( (f * g) * h \big) \circ i] = [\big( (f * g) *
  h \big) \circ T ] = [f * (g * h)] \]
\end{proof}
