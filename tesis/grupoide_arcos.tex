\subsection{Grupoide de arcos}
Si tenemos dos arcos continuos \(f,g\) tales que el punto final de \(f\)
sea el punto inicial de \(g\), podemos construir un arco continuo que
recorra \(f\) y luego recorra \(g\). Esta idea es formalizada como
el producto de dos arcos.

\begin{definicion}[Producto de arcos] \label{def:prod-arcos}
Para dos arcos \(f,g : I \to X\) tales que
\(f(1) = g(0)\), se define el producto \(f * g \)
\[ (f*g) (s) \coloneqq \begin{cases}
    f(2s) & s \in [0,\frac{1}{2}] \\
    g(2s - 1) & s \in [\frac{1}{2} , 1]
  \end{cases}
\]
la cual sigue siendo una funcion continua en virtud del \emph{lema del
pegamiento}.
\end{definicion}

Esta construccion se puede reutilizar para clases
\emph{arco}-homotopicas \([f],[g]\) que compartan punto final e inicial
respectivamente.
\begin{definicion}[Producto de clases Arco-homotopicas]
  Sean \([f],[g]\) dos clases arco-homotópicas tales que \( f(1) =
  g(0)\). Se define el producto de clases \([f] * [g]\) por la ecuación
  \[ [f] * [g] \coloneqq [f * g] \]
  con el segundo producto \(f * g\) siendo el producto de arcos de la
  definición \ref{def:prod-arcos}.
\end{definicion}
\begin{acotacion}
  Este producto esta bien definido como clase homotopica. Esto se ve
  porque la única manera en que este no lo estuviera es si existieran las
  relaciones \(\hat f \simeq f\) y \(\hat g \simeq g\) con \(f(1) = g(0)\)
  que cumplieran que \(\hat f * \hat g \not \in [f * g]\). Esto no ocurre,
  pues por definición existen homotopías \(F,G\) respectivas para \(\hat f
  \simeq f\) y \(\hat g \simeq g\) que cumplen
  \begin{equation}
  F(1,t) = G(0,r) , \quad \forall t,r \in [0,1] \label{eq:hom-same-point}
  \end{equation}
  y por tanto, podemos definir una nueva homotopía de \(\hat f * \hat
  g \simeq f * g\) por
  \[
    H(s,t) = \begin{cases}
      F(2s,t) & s \in [0, \frac{1}{2}] \\
      G(2s - 1, t) & s \in [\frac{1}{2} , 1]
    \end{cases}
  \]
  la cual es bien definida en virtud de \eqref{eq:hom-same-point} y
  continua por el \emph{lema del pegamiento}.
\end{acotacion}

\paragraph{} Con esta operación binaria, una pregunta natural es si
\((\mathcal C (I , X)/_\simeq , (*))\), esta es la clase de los arcos
sobre un espacio topológico \(X\), tiene estructura de grupoide. Esto
equivale a cumplir las siguientes 3 propiedades
\begin{description}
\item[1. Asociatividad:] Si \([f] * ([g] * [h])\) esta definido entonces
  tambien lo debe estar \(([f] * [g]) * [h]\) y ademas deben coincidir.
\item[2. Identidades izquierda y derecha:] Para todo \([f]\) con
  \(x_0, x_1\) puntos inicial y final respectivamente, debe de
existir elementos \([k_{x_0}], [k_{x_1}]\) tales que
\[ \begin{matrix}
    [f] * [k_{x_1}] = [f] & & [k_{x_0}] * [f] = [f]
  \end{matrix}
\]
\item[3. Inverso:] Para todo \([f]\) clase de arcos con \(x_0, x_1\)
  puntos inicial y final respectivamente debe de existir un elemento
  \([f^{-1}]\) que cumpla
\[ \begin{matrix}
    [f] * [f^{-1}] = [k_{x_0}] & & [f^{-1}] * [f] = [k_{x_1}]
  \end{matrix}
\]
\end{description}
Para probar esto necesitamos primero definir a nuestros candidatos de
\(k_{x_0}, k_{x_1}, f^{-1}\) ademas de algunas funciones auxiliares.
Iniciando por los mapeos constantes e identidad en \(I \to I\)
\[
  \begin{matrix}
     e_0 :     & I \to I; \\
     e_1 :     & I \to I; \\
     i :       & I \to I; \\
     \bar{i} : & I \to I;
   \end{matrix}
   \quad
   \begin{matrix}
      e_0(t) &\coloneqq &0 \\
      e_1(t) &\coloneqq &1 \\
      i(t)   &\coloneqq &t   \\
      i(t)   &\coloneqq &1 - t
   \end{matrix}
\]
Ademas, para todo arco \(f : I \to X \), el elemento \(f^{-1} : I \to X \) esta
definido (en el espiritu de la ecuacion \eqref{eq:homotopy-simetry}) por
\[ f^{-1} (s) \coloneqq f (1 - s) \]
El cual, si \(f\) es un arco de \(x_0 \to x_1\) entonces \(f^{-1}\) es
un arco de \(x_1 \to x_0\). Por ultimo, para todo \(x \in X \) se define
la curva constante
\begin{align*}
  k_x : &I \longrightarrow X \\
        &t \longmapsto x
\end{align*}
Con esto podemos afrontar la demostración
\begin{proof}
En el orden de las demostraciones, se procedera en mostrando las
identidades izquierdas y derechas (2), luego la existencia del inverso
(3) y se finaliza con la asociatividad (1). Ademas se probaran lemas
intermedios en cada item de ser necesario. Sea \(f : I
\to X\) el representante de la clase \([f]\). sean \(x_0, x_1\) los
puntos inicial y final de \(f\) respectivamente.

\paragraph{(2).} Dados \(i,e_1\) arcos en \(I\) definidos anteriormente,
es claro que \(i * e_1\) es también un arco (continuo) en \(I\); mas aun,
ya que \(I = [0,1]\) es convexo, se tiene \(i * e_1 \simeq i\) con la
homotopía de la linea recta entre estos denotada por \(C\).

También conocemos que la composición de funciones continuas es
continua, de lo cual podemos decir que dada una homotopía \(H : I \times
I \to I\) la composición \( f \circ H : I \times I \to X\) es una homotopía.

Por otro lado, necesitamos conocer como se comporta la composición con
respecto a nuestro producto \((*)\), para esto tenemos el siguiente lema
\begin{lema}[Distributividad de la composicion sobre producto]
\label{lema:dist-composición-producto}
\[\forall a,b : I \to I,\ \forall f : I \to X,\ f \circ (a * b) = (f
\circ a) * (f \circ b) \]
\end{lema}
\begin{proof}
  \[ f \circ (a*b) (s) =
    \begin{cases}
      f (a(2s)) & s \in [0,\frac{1}{2}] \\
      f (b(2s - 1)) & s \in [\frac{1}{2} , 1]
    \end{cases}
    = (f \circ a) * (f \circ b) (s)
  \]
\end{proof}

Luego esto nos permite afirmar en especifico que
\[ f \circ i \simeq f \circ (i * e_1) = (f \circ i) * (f \circ e_1) \]
en virtud de la homotopía
\[ f \circ C : I \times I \to X \]
Por tanto
\begin{equation}\label{eq:homequiv2.1}
[f \circ i] = [f \circ (i * e_1)] = [(f \circ i) * (f \circ e_1)]
\end{equation}
Si tomamos este ultimo termino, sabemos que por definición
\begin{equation}\label{eq:homequiv2.2}
[(f \circ i) * (f \circ e_1)] = [(f \circ i)] * [(f \circ e_1)]
\end{equation}
y es claro también que
\[
  \begin{matrix}
    f \circ i = f & f \circ e_1 = k_{x_1} \\
  \end{matrix}
\]
\begin{equation}\label{eq:homequiv2.3}
[(f \circ i)] * [(f \circ e_1)] = [f] * [k_{x_1}]
\end{equation}
juntando
\eqref{eq:homequiv2.1},\eqref{eq:homequiv2.2},\eqref{eq:homequiv2.3} nos
muestra que \([k_{x_1}]\) es la identidad por la derecha. Podemos
hacer un proceso análogo para \([k_{x_0}]\) como identidad por la
izquierda, así probando (2).

\paragraph{(3).} De manera similar a la pregunta anterior, notar que \(i,
\bar{i}\) definidos anteriormente cumplen
\[ i * \bar{i} \simeq e_0 \]
por la homotopía de linea recta en el espacio \(I\) que es convexo.
Utilizando el lema \eqref{lema:dist-composición-producto} tenemos que
\[ [f * f^{-1}] = [f \circ (i * \bar{i})] = [f \circ e_0] = [k_{x_0}] \]
\[ \implies [f] * [f^{-1}] = [k_{x_0}] \]
De manera análoga se prueba la inversa por la izquierda.

\paragraph{(1).}
Por definición, tenemos que las diferentes asociaciones nos resultan en
diferentes representantes de clase
\[
  \begin{matrix}
    (f * g) * h \, (t) =
    \begin{cases}
      f (4t), & t \in [0, \frac 1 4] \\
      g (4t - 1), & t \in [\frac 1 4, \frac 1 2] \\
      h (2t - 1), & t \in [\frac 1 2, 1] \\
    \end{cases} &
    f * (g * h) \, (t) =
    \begin{cases}
      f (2t), & t \in [0, \frac 1 2] \\
      g (4t - 2), & t \in [\frac 1 2, \frac 3 4] \\
      h (4t - 3), & t \in [\frac 3 4, 1] \\
    \end{cases}
  \end{matrix}
\]
Sea \(T : I \to I\) definida por
\[ T (t) :=
  \begin{cases}
    2t & t \in [1 , \frac 1 4] \\
    t + \frac 1 4 & t \in [\frac 1 4, \frac 1 2] \\
    \frac t 2 + \frac 1 2 & t \in [\frac 1 2, 1]
  \end{cases}
\]
\(T\) es una función continua por el lema del pegamiento. Por ser \(I\)
un dominio convexo, se tiene que \(i \simeq T\), luego
\[ [(f * g) * h] = [\big( (f * g) * h \big) \circ i] = [\big( (f * g) *
  h \big) \circ T ] = [f * (g * h)] \]
\end{proof}
