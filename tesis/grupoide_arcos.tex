\subsection{Grupoide de arcos}
Si tenemos dos arcos continuos \(f,g\) tales que el punto final de \(f\)
sea el punto inicial de \(g\) nos da la idea de que podemos construir un
camino continuo que recorra \(f\) luego recorriendo \(g\). Esta idea es
conocida formalmente como el producto de dos arcos.

\begin{definicion}[Producto de arcos]
Para dos arcos \(f,g : I \to X\) tales que
\(f(1) = g(0)\), se define el producto \(f * g \)
\[ (f*g) (s) = \begin{cases}
    f(2s) & s \in [0,\frac{1}{2}] \\
    g(2s - 1) & s \in [\frac{1}{2} , 1]
  \end{cases}
\]
la cual sigue siendo una funcion continua en virtud del lema del
pegamiento.
\end{definicion}

Esta construccion se puede reutilizar para clases
\emph{arco}-homotopicas \([f],[g]\) que compartan punto final e inicial
respectivamente para definir un producto de clases de equivalencia
\[ [f] * [g] := [f * g]\]
El cual esta bien definido pues si \(f \simeq_{ah} f'\), \(g \simeq_{ah}
g'\) a traves de \(F, G\) respectivamente
\[H(s,t) = \begin{cases}
    F(2s,t) & s \in [0, \frac{1}{2}] \\
    G(2s - 1, t) & s \in [\frac{1}{2} , 1]
  \end{cases}
\]
Es la homotopia que relaciona a los arcos que sean homotopicos a
\(f*g\). Esta ademas es continua en virtud otra vez del lema del pegamiento.

Con esta operacion binaria, una pregunta a hacerse es si tiene
\((X/\simeq_{ah} , *)\) posee
estructura de grupoide. Esto equivale a cumplir 3 propiedades
\begin{enumerate}
\item \textbf{Asociatividad} Si \([f] * ([g] * [h])\) esta definido entonces
  tambien lo debe estar \(([f] * [g]) * [h]\) y deben de ser iguales.
\item \textbf{Identidades izquierda y derecha} Para todo \([f]\) con
  \(x_0, x_1\) puntos inicial y final respectivamente, debe de
existir elementos \([k_{0}], [k_{1}]\) tales que
\[ \begin{matrix}
    [f] * [k_{1}] = [f] & & [k_{0}] * [f] = [f]
  \end{matrix}
\]
\item \textbf{Inverso} Para todo \([f]\) clase de arcos con \(x_0, x_1\)
  puntos inicial y final respectivamente debe de existir un elemento
  \([\bar{f}]\) que cumpla
\[ \begin{matrix}
    [f] * [\bar{f}] = [k_{x_0}] & & [\bar{f}] * [f] = [k_{x_1}]
  \end{matrix}
\]
\end{enumerate}
Para probar esto necesitamos primero definir a nuestros candidatos de
\(k_{x_1}, k_{x_2}, \bar{f}\) ademas de algunas funciones auxiliares,
iniciando por los mapeos constantes e identidad en \(I\)
\[ \begin{matrix}
     e_0 : & I \to I & e_0(t) := 0 \\
     e_1 : & I \to I & e_1(t) := 1 \\
     i :   & I \to I & i(t) := t \\
     \bar{i} : & I \to I & i(t) := 1 - t
   \end{matrix}
   \]
Para todo arco \(f : I \to X \), el elemento \(\bar{f} : I \to X \) esta
definido (en el espiritu de \eqref{eq:homotopy-simetry}) por
\[ \bar{f} (s) := f (1 - s) \]
Para todo \(x \in X \) se define la curva constante
\begin{align*}
  k_x : &I \to X \\
        &t \mapsto x
\end{align*}
Con esto podemos afrontar la demostración
\begin{proof}
Se procedera \(2 \to 3 \to 1\) con lemas en medio. Sea \(f : I \to X\) el
representante de \([f]\). sean \(x_0, x_1\) los puntos inicial y final de
\(f\) respectivamente.

\paragraph{(2).} Dados \(i,e_1\) arcos en \(I\) definidos anteriormente,
es claro que \(i * e_1\) es tambien un arco (continuo) en \(I\); mas aun,
ya que \(I\) es convexo se tiene \(i * e_1 \simeq_{ah} i\) con la
homotopia de la linea recta entre estos. Se desprende de esto ultimo que
\( [i] = [i * e_1]\). Para continuar necesitamos el siguiente lema
\begin{lema}[Distributividad de la composicion sobre producto.]
\label{lema:dist-composicion-producto}
\[\forall a,b : I \to I,\ f \circ (a * b) = (f \circ a) * (f \circ b) \]
\end{lema}
\begin{proof}
  \[ f \circ (a*b) (s) =
    \begin{cases}
      f (a(2s)) & s \in [0,\frac{1}{2}] \\
      f (b(2s - 1)) & s \in [\frac{1}{2} , 1]
    \end{cases}
    = (f \circ a) * (f \circ b)
  \]
\end{proof}
Recordemos ademas que la composición de funciones continuas es continua.
Luego esto nos permite afirmar en especifico que
\[ f \circ (i * e_1) \simeq_{ah} (f \circ i) * (f \circ e_1) \] en virtud
de la composición de \(f\) con la homotopia de linea recta entre \(i *
e_1 \) y \(i\). Obteniendo asi que
\[ [f \circ (i * e_1)] = [(f \circ i) * (f \circ e_1)] \] pero al reducir
las composición obtenemos
\[
  \begin{matrix}
    f \circ i = f & f \circ e_1 = k_{x_1} & f \circ (i * e_1) = f *
    k_{x_{1}} \\
  \end{matrix}
\]
\[ \implies [f] * [k_{x_1}] := [f * k_{x_1}] = [f \circ (i * e_1)] = [ f
  \circ i] = [f] \]
probando (2).

\paragraph{(3).} De manera similar a la pregunta anterior, notar que \(i,
\bar{i}\) definidos anteriormente cumplen
\[ i * \bar{i} \simeq_{ah} e_0 \]
por la homotopia de linea recta en el espacio \(I\) que es convexo.
Utilizando el lemma \eqref{lema:dist-composicion-producto} tenemos que
\[ f * \bar{f} = f \circ (i * \bar{i}) \simeq_{ah} f \circ e_0 =
  k_{x_0} \]
\[ \implies [f] * [\bar{f}] = [k_{x_0}] \]
De manera analoga se prueba la inversa por la izquierda.

\paragraph{(1).}
% todo(slack)
\end{proof}
