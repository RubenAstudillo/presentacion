\subsection{Cubrimientos}
\begin{definicion}
Sea \(p : \tilde{X} \to X\) una función continua sobreyectiva. El
abierto \(U \subseteq X\) se dice \textbf{cubierto} por \(p\)
si existe \(\{V_\alpha\}_{\alpha \in \Lambda},\ \Lambda \subseteq
\mathbb N\) tal que
\[ p^{-1} (U) = \bigcup_{\alpha \in \Lambda} V_\alpha \]
Donde \(\{V_\alpha\}\) es una familia disjunta de abiertos, tal que
\(\forall \alpha \in \Lambda,\ p \mid_{\alpha}\) es homeomorfo sobre \(U\).
Si para todo \(x \in X\) existe dicha vecindad \(U\) que cumpla lo
anterior, se dirá que \(\tilde{X}\) es un \textbf{espacio cubrimiento}
de \(X\) y el par \((p,\tilde X)\) denota el cubrimiento para \(X\).
\end{definicion}

Un ejemplo simple de cubrimiento para \(S^1\) es \(p :
\mathbb R \to S^1,\ p(t) := e^{2 \pi \imath t}\). Este es claramente
sobreyectivo y continuo, escogiendo dos abiertos ejemplares como \(U =
S^1 - \{(1,0)\}\) y \(V = S^1 - \{(-1,0)\}\), es claro que
\[
    p^{-1} (U) = \bigcup_{n \in \mathbb Z} (n, n+1)
    \qquad p^{-1} (V) = \bigcup_{n \in \mathbb Z} (n - \frac 1 2, n + \frac 1
    2 )
\]
Donde estas familias de conjuntos son disjuntos y restringidos a cada
uno se tiene la inactividad requerida.

A priori el cubrimientos es independiente de los caminos que tengamos
en \(X\), queremos ver si podemos reflejar la información importante de
los caminos de \(X\) sobre \(\tilde{X}\).
\begin{figure}[h]
  \centering
  \includegraphics[scale=0.3]{./imagenes/lifting-path.png}
\end{figure}
Es decir, para todo arco \(f : I \to X\), construiremos \(\tilde f : I
\to \tilde X\) el cual cumpla \(p \circ \tilde f = f \), donde \(\tilde
f\) sera nuestro representante en el espacio cubrimientos de estos
caminos. Veremos también que bajo ciertas hipótesis, este \(\tilde f\)
es único y refleja información homotopica. Para esto necesitamos
desarrollar algunos teoremas previos.
\begin{lema}[Numero de Lebesgue] \label{thm:lebesgue-number-lema}
  Sea \(\mathcal A\) un cubrimiento del espacio métrico \((X,d)\). Si
  \(X\) es compacto, entonces existe \(\delta > 0\) tal que para todo
  subconjunto de \(X\) teniendo diámetro menor que \(\delta\), existe un
  elemento de \(\mathcal A\) conteniéndolo.
\end{lema}
\begin{proof}
  Supongamos que \(X \not \in \mathcal A\), pues si no trivialmente el
  teorema se cumple \(\forall \delta > 0\). Por compacidad de \(X\)
  existe una colección \(\{A_1,\dotsc,A_n\} \subset \mathcal A\) que
  cubre a \(X\), definamos a los conjuntos \(C_i = X - A_i,\ \forall i
  \in [1,n]\) y a la función \(f : X \to \mathbb R\) definida por
  \[ f(x) := \frac 1 n \sum_{i=1}^{n} d(x, C_i) \]
  i.e. la distancia promedio de \(x\) a \(C_i\). Notemos que \(\forall x
  \in X,\ f(x) > 0\), pues para \(x \in A_i \subseteq X\), por ser
  \(A_i\) un abierto, existe \(\epsilon > 0\) tal que \(B(x,\epsilon)
  \subset A_i\) y por tanto \(d(x, C_i) \geq \epsilon\) que implica \(
  f(x) \geq \frac \epsilon n > 0\).

  Por otro lado \(f\) es una función continua sobre \(X\) un espacio
  compacto, por lo tanto alcanza un mínimo; a este le denotaremos como
  nuestro \(\min_X f \coloneqq \delta \).

  Probaremos que este \(\delta\) cumple el requerimiento, sea \(B
  \subset X\) subconjunto abierto de diámetro menor que \(\delta\), sea
  \(x \in B\) arbitrario. Definamos escojamos el conjunto \(C_m\) como
  \[ C_m := \max_{i \in [1,n]} d(x, C_i) \]
  Dado que
  \[\delta \leq f(x) \leq d(x, C_m) \]
  Esto nos dice que dado \(\delta \leq d(x, C_m)\), existe una vecindad
  de al menos diámetro \(\delta\) que contiene a \(x\) en \(X - C_m = A_m\).
\end{proof}
\begin{definicion}[Levantamiento de \(f\)]
  Sea \(p : \tilde X \to X\) un mapeo. Si \(f : W \to X\) es un mapeo
  continuo, un levantamiento de \(f\) es una función \(\tilde f : W \to
  \tilde X\) tal que \(p \circ \tilde f = f\).
\end{definicion}
Ver que la definición es mucho mas general que lo que pedíamos al
diagrama. Usualmente \(W = [0,1]\) pues estudiaremos los caminos sobre
\(X\). Veremos a continuación teoremas de como se reflejan los caminos
y las homotopias en el espacio cubrimiento de \(X\).
\begin{teorema}\label{thm:lifting-theorem}
  Sea \(p : \tilde X \to X\) un cubrimiento y \(x_0 \in X\). Fijemos
  algún \(\tilde x _0 \in \tilde X\) tal que \(p(\tilde x _0) = x_0 \).
  Para cualquier camino \(f : [0,1] \to X\) que comience en \(x_0\), existe
  un único camino levantamiento \(\tilde f : [0,1] \to \tilde X\) tal que
  \(\tilde f (0) = \tilde x _0\)
\end{teorema}
\begin{proof}
  Para todo punto de \(x \in X\), existe una vecindad \(U_x\) que es
  cubierta, por tanto \( X \subseteq \bigcup_{x \in X} U_x\) para cada
  \(U_x\) correspondiente.
  Por otro lado, tenemos que \( [0,1] \subset f^{-1} (\bigcup_{x
  \in X} U_x)\); por el Teorema \ref{thm:lebesgue-number-lema}, podemos
  elegir \(s_0,\dotsc,s_n \in [0,1]\) tal que \(f([s_i, s_{i+1}])\) este contenido
  en algún \(U_x\). Definiremos \(\tilde f\) inductivamente.

  Primero declaremos \(\tilde f (0) = \tilde x _0\). Luego, suponiendo
  que \(\tilde f (s)\) esta definido para \(s \in [0, s_i]\), se
  define a \(\tilde f \) en \([s_i, s_{i+1}]\) de la siguiente forma.
  Dado que para algún \(x \in X\),
  \[f ([s_i, s_{i+1}]) \subseteq U_x \quad \land \quad \exists
    \{V_\alpha\}_{\alpha \in \Lambda},\ \bigcup_{\alpha \in \Lambda}
    V_\alpha = p^{-1} (U_x)\]
  debe de existir \(\alpha \in \Lambda\) tal que \(\tilde f (s_i) \in
  V_\alpha\) previamente definido; a este, le denotaremos \(V_0\). Dado
  que \(p \mid_{V_0}\) es un homeomorfismo, definimos a \(\tilde f (s)\)
  en \([s_i, s_{i+1}]\) por
  \begin{equation}
  \tilde f (s) = (p \mid _{V_0})^{-1} (f(s))\label{eq:tilde-f-inductiva}
  \end{equation}
  El cual es continuo en \([s_i, s_{i+1}]\) en virtud del lema del
  pegamiento y bien definido por ser \((p \mid _{V_0})^{-1} \) homeomorfismo.

  Para ver la unicidad, se probara inductivamente. Supongamos que existe
  otro \(\hat{f}\) levantamiento par de \(f\) que también
  comienza en \(x_0\), ie \(\hat{f} (0) = \tilde x _0 = \tilde f
  (0)\). Supongamos que que \(\forall s \in [0, s_i],\ \hat{f}
  (s) = \tilde f (s)\), dado que \(\tilde f\) esta definida por
  \eqref{eq:tilde-f-inductiva}, \(\hat f\) debe ser
  eventualmente diferente a la definición \eqref{eq:tilde-f-inductiva},
  pero
  \[\hat f (s_i) = \tilde f (s_i) \in V_0\]
  con \([s_i, s_{i+1}]\) conexo y la familia \(\{V_\alpha\}\)
  es disjunta, obliga\footnote{Funciones continuas mapean conjuntos
    conexos a conexos} a que \(\hat f [s_i, s_{i+1}] \subset
  V_0\). Dado que es un levantamiento, debe de cumplirse que
  \[\forall s \in [s_i, s_{i+1}],\ p \circ \hat f \, (s) =
    f(s) = p \circ \tilde f \, (s) \]
  \[ \iff p ( \hat f (s)) = p ((p \mid_{V_0})^{-1} (f (s)))\]
  \[ \implies \hat f (s) = (p \mid_{V_0})^{-1} (f (s)), \quad \forall s
      \in [s_i, s_{i+1}]\]
  siendo esta la única posible definición, pues de haber otra,
  se tendria una contradiccion en decir que \((p \mid_{V_0})\) es un
  homeomorfismo (mapeo único), por tanto se obliga a que \( \hat f =
\tilde f\)
\end{proof}
Mas aun, podemos no solo levantar las curvas sobre \(X\) a \(\tilde X\),
si no también preservar las homotopias de \(X\) en \(\tilde X\).
\begin{corolario}
  Sea \(p : \tilde X \to X\) un cubrimiento par tal que \(p(\tilde x _0)
  = x_0 \) para algún \(\tilde x _0 \in \tilde X\). Para cualquier
  función continua \(F : I \times I \to X\) tal que \(F(0,0) = x_0\), tiene una
  única función levantamiento continua \(\tilde F : I \times I \to
  \tilde X\) que cumpla \(\tilde F (0,0) = \tilde x_0\)
\end{corolario}
% Pagina 24 quitar referencias a la palabra par
\begin{proof}
  Lo único diferente con el teorema \ref{thm:lifting-theorem} es que
  tomamos \(I \times I\) que sigue siendo compacto. Podemos aplicar el
  teorema anterior primero definiendo \(\tilde F\) en \(0 \times I\),
  luego en \(I \times 0\) y escogiendo subdivisiones de \(I \times I\)
  \[ s_0 < s_1 < \dotsc < s_m \]
  \[ t_0 < t_1 < \dotsc < t_n \]
  Tales que \(F ([s_i , s_{i+1}] \times [t_j \times t_{j+1}])\) este
  contenido en un conjunto cubierto en \(X\), esto utilizando
  el lema de Lebesgue al igual que en la demostración anterior.

  Luego definimos inductivamente \(\tilde F\) sobre \([s_i, s_{i+1}]
  \times [t_j , t_{j+1}]\) siempre y cuando \(\tilde F\) ya este
  definida sobre las lineas
  \[ L := [s_i , s_{i+1}] \times \{t_j\}\]
  \[ H := \{s_i\} \times [t_j , t_{j+1}] \]
  que son los bordes inferior y izquierdos del rectángulo.

  Tomamos el conjunto \(U\) par que contiene a la imagen
  \[ F([s_i , s_{i+1}] \times [t_j , t_{j+1}]) \subseteq U \subset X \]
  Por su paridad, podemos tomar la pre-imagen de \(p\) en \(U\) como
  \[ p^{-1}(U) = \bigcup_{\alpha \in \Alpha} V_\alpha\]
  Donde la familia \(\{V_\alpha\}\) es disjunta.

  Dado que \( \{(s_i, t_j)\} = L \cap H\) y que \( (s_i , t_j) \) ya
  esta definido bajo \(\tilde F\), existe un único \(V_0 \in
  \{V_\alpha\}\) que contiene a \(F (s_i, t_j)\). Pero notando que
  \(\{V_\alpha\}\) son disjuntos y que \(L,H\) son conexos, obliga a que
  \[ \tilde F (H) \subset V_0 \quad \tilde F (L) \subset V_0 \]
  Luego podemos definir \(\tilde F\) sobre el resto de \([s_i,
    s_{i+1}] \times [t_j , t_{j+1}]\) notando que \(p \mid_{V_0}\) es
  un homeomorfismo entre \(V_0\) y \(U\) y por tanto
  \[ \forall (a,b) \in [s_i, s_{i+1}] \times [t_j , t_{j+1}],
    \ \tilde F (a,b) := p^{-1} \mid_{V_0} (F(a,b)) \]
  Esta cumple la regla \( F = p \circ \tilde F\) y es continua en virtud
  del lema del pegamiento. Notando finalmente que ahora los conjuntos
  \[ L := [s_i , s_{i+1}] \times \{t_{j+1}\}\]
  \[ H := \{s_{i+1}\} \times [t_j , t_{j+1}] \]
  Están definidos en \(\tilde F\) y pueden ser utilizados para seguir
  definiéndola inductivamente sobre \(I \times I\).
%fin pagina 24
\end{proof}

Ahora es mas clara la forma de calcular grupos fundamentales usando
cubrimiento. Si tuviéramos de partida un isomorfismo de grupos
entre los espacios \(\pi (\tilde X)\) y \(\pi (X)\), tendríamos que
presuponer la estructura del grupo en \(\pi (\tilde X)\). La alternativa
presentada muestra que los cubrimientos, preservan los arcos y
homotopias de \(X\) en \(\tilde X\) y si es el espacio \(\tilde X\) es
suficientemente simple como para admitir un calculo simple de su grupo
fundamental, entonces podemos saber que el grupo de \(X\) es un subgrupo
normal del grupo de \(\tilde X\).

Para un calculo mas explicito necesitamos denotar los puntos finales
obtenidos en \(\tilde X\).
\begin{definicion}[Levantamiento derivado]
  Sea \(p : \tilde X \to X\) un cubrimiento y sea \(x_0 \in X\).
  Escojamos \(\tilde x _0 \in \tilde X\) tal que \(p(\tilde x _0) =
  x_0\). Dado un elemento \([f] \in \pi (X, x_0)\), sea \(\tilde f\) el
  levantamiento (único) de \(f\) a un camino en \(\tilde X\) que
  comienza en \(\tilde x _0\). Definamos \(\phi ([f]) = \tilde f (1)\) del
  \(\tilde f\) asociado a \(f\) anteriormente. Entonces \(\phi\) es bien
  definido como mapeo de conjunto
  \[ \phi : \pi (X, x_0) \longrightarrow p^{-1} (x_0)\]
  llamando a \(\phi\) el \textbf{correspondiente levantamiento
  derivado}\footnote{Corresponding lifting path} del cubrimiento \(p\).
\end{definicion}
La definición anterior intuitivamente ``desenrolla'' los ciclos del
grupo fundamental \(\pi (X,x_0)\) y ve donde terminan vistos sobre
\(\tilde X\).
\begin{teorema}
  Sea \(p : \tilde X \to X\) un cubrimiento tal que \(p (\tilde x _0) =
  x_0\). Si \(\tilde X\) es arco-conexo entonces el correspondiente
  levantamiento \(\phi : \pi (X, x _0) \to p^{-1} (x_0)\) es
  sobreyectivo. Mas aun, si \(\tilde X\) es simplemente conexo, este es
  biyectivo.
\end{teorema}
Usualmente queremos que \(p^{-1} (x)\) no sea simplemente un conjunto,
queremos poder dotarle de estructura de grupo. En caso de \(\tilde X\)
ser arco-conexo, lo que estamos haciendo realmente es estudiar un
subgrupo de \(\pi \left( X, x_0 \right)\) a través de \(p^{-1}(x_0)
\subseteq \tilde X\).
\begin{proof}
  Demostración estándar de sobreyectividad en caso de ser \(\tilde X\)
  arco-conexo. Para \(\tilde x _1 \in p^{-1} (x_0)\) arbitrario, existe
  \(\tilde f\) camino entre \(\tilde x _0 \) a \(\tilde x _1\) por
  arco-conexidad. Definimos \(f := p \circ \tilde f\), el cual es un
  camino \(f : I \to X\) que cumple \(\phi ([f]) = \tilde x _1\) por
  definición.

  En caso de ser \(\tilde X\) simplemente-conexo, veremos solo la
  inyectividad. Sean \([f],[g] \in \pi (X, x_0)\) tales que \(\phi([f])
  = \phi([g])\), mostraremos que \([f] = [g]\). Sean \(\tilde f, \tilde
  g : I \to \tilde X\) los levantamientos respectivos que comienzan en
  \(\tilde x _0\). Dado que \(\phi([f]) = \tilde f (1) = \tilde g (1) =
  \phi([g])\) y la simple-conexidad de \(\tilde X\), existe \(\tilde F :
  I \times I \to \tilde X\) homotopía \(\tilde f, \tilde g\). Luego \(p
  \circ \tilde F : I \times I \to X \) es una homotopía entre \(f, g\),
  por tanto \([f] = [g]\)
\end{proof}
Ahora podemos plantearnos un ejemplo clásico, ver el grupo fundamental
de \(S^1\). Utilizaremos el teorema anterior para proponer un
cubrimiento tal que \(\tilde X\) sea simplemente conexo y que pueda
preservar la estructura de grupo para ser estudiado bajo ese lente.
\begin{teorema}
  \(\forall x_0 \in S^1,\ \big( \pi (S^1,x_0), * \big)\) es isomorfo a
  \((\mathbb Z, +)\)
\end{teorema}
\begin{proof}
  Dado que \(S^1\) es arco-conexo, basta probar que este teorema es
  cierto para algún \(x_0\) en \(S^1\). Sea \(p : \mathbb R \to S^1,\
  p(t) := (\cos 2 \pi t, \sin 2 \pi t)\) el cual es fue visto anteriormente
  que es un cubrimiento par. Sea \(\tilde x _0 = 0\) y \( p(\tilde x _0) =
  (1,0) =: x_0 \in S^1\). Por ser \(\mathbb R\) simplemente conexo, se
  tiene que el levantamiento correspondiente \(\phi : \pi (S^1, x_0) \to
  p^{-1} (x_0)\) es biyectivo, donde
  \[ p^{-1} (1,0) = \{t \in \mathbb R \mid (\cos 2 \pi t, \sin 2 \pi t)
    = (1, 0) \} = \mathbb Z \]
  Notamos además que de sobre todo \(\mathbb Z\) existe únicamente el
  grupo bajo la adición, este sera nuestro grupo de partida.

  Para probar que es homomorfismo, se toman \([f], [g] \in \pi
  (S^1, x_0)\) arbitrarios y \(\tilde f, \tilde g\) sus correspondientes
  levantamientos tales que \(n := \tilde f (1) = \phi ([f]),\ m :=
  \tilde g (1) = \phi ([g])\). Ahora queremos encontrar quien es el
  levantamiento de \([f] * [g]\), para esto se define el camino
  \begin{align*}
    \tilde{\tilde g} : I &\longrightarrow \mathbb R \\
    s &\longmapsto n + \tilde g (s)
  \end{align*}
  Puesto que \(p(n + x) = p(x)\) por periodicidad, se cumple que
  \[ p \circ \tilde{\tilde g} (x) = p (n + \tilde g (x)) = p (\tilde g
    (x)) = g (x) \]
  Por tanto \(\tilde{\tilde g}\) es el levantamiento de \(g\) para
  \(\tilde x_0 = n\) (en vez de \(\tilde x_0 = 0\)). Luego el producto
  \(\tilde f * \tilde{\tilde g}\) esta bien definido pues \(\tilde f (1)
  = n = \tilde{\tilde g} (0)\) y se afirma que este es el levantamiento
  de \(f * g\) pues por calculo
  \[ p \circ (\tilde f * \tilde{\tilde g}) =
     ((p \circ \tilde f) * (p \circ \tilde{\tilde g})) =
     (f * g)
  \]
  Por ultimo calculamos
  \[ \phi ([f] * [g]) = \tilde{\tilde g} (1) = n + m = \phi ([f]) + \phi
  ([g]) \]
  Por lo que \(\phi\) es un homomorfismo de grupos.
\end{proof}

\paragraph{Ejemplo: Cubrimiento \(p_n : S^1 \to S^1\).} Para testear la intuición
de que el espacio cubrimiento refleja en el caso común un subgrupo del
espacio topológico, tomemos el caso
\begin{align*}
  p_n : S^1 &\longrightarrow S^1 \\
  z &\longmapsto z^n
\end{align*}
Para este, nuestro conjunto \(p^{-1} (x_0)\), con \(x_0 = (1,0)\) se
obtiene mediante la reducción
\begin{align*}
  p^{-1} (1,0)
    &= \{ z \in S^1 \mid z^n = (1,0)\} \\
    &= \{z \in S^1 \mid (\cos 2 \pi n t, \sin 2 \pi n t) = (1,0) :
           t \in [0,1] \} \\
    &= \{(\cos 2 \pi t, \sin 2 \pi t) \in S^1 \mid
           t \in \{0, \frac 1 n , \dotsc , \frac {n-1} n \} \}
\end{align*}
\[ \therefore p^{-1}(1,0) \simeq \mathbb Z / n \mathbb Z \]
Ademas bajo el mismo tratamiento utilizado anteriormente en \(\pi (S^1,
x_0) = \mathbb Z\) para \(\phi\), este es un homo-morfismo. Esto muestra
que la eleccion de \(p\) afecta que subgrupo de \(\pi (S^1, x_0)\) estudiaremos.
\begin{teorema}
  Sea \(p : \tilde X _1 \to X_1,\ p' : \tilde X _2 \to X_2 \) dos
  cubrimientos pares, entonces
  \[ p_1 \times p_2 : \tilde X _1 \times \tilde X _2 \to X_1 \times X_2 \]
  Es un cubrimiento par.
\end{teorema}
\begin{proof}
  Para todo \(x_1 \in X_1,\ x_2 \in X_2\), existen \(U_1, U_2\)
  vecindades pares respectivamente tales que son cubiertas por \(p_1,
  p_2\). Denotemos a \(\{V_\alpha^{(1)}\}, \{V_\beta^{(2)}\}\) familias
  de conjuntos tales que
  \[
    \begin{matrix}
      \bigcup_{\alpha} V_\alpha^{(1)} = p_1^{-1} (U_1) &
      \bigcup_{\alpha} V_\alpha^{(2)} = p_2^{-1} (U_2)
    \end{matrix}
  \]
  \[ \bigcup_{\alpha, \beta} V_\alpha^{(1)} \times V_\beta^{(2)} = (p_1
    \times p_2)^{-1} (U_1 \times U_2)\]
  Donde la restriccion a un \(V_{\bar{\alpha}}^{(1)} \times
  V_{\bar{\beta}}^{(2)}\) arbitrario en \(p_1 \times p_2\) obtiene un
  homeomorfismo en virtud de las componentes.
\end{proof}

Esto es una extension del resultado que ya conociamos sobre el functor
entre \(\mathbf{HoTop}_{*}\) y \(\mathbf{Grp}\) en cuanto a preservacion
del producto. Esto no dice que el ``calculo'' de grupos fundamentales
tambien puede derivarse del conocimiento del productor de cubrimientos
pares.

\paragraph{Ejemplo Toro en \(R^4\).} Dado que podemos caracterizar al
toro \(T^1\) como \(S^1 \times S^1\), el teorema anterior nos dice que
\((\pi (T^1, b), *) = (\pi (S^1, b) \times \pi (S^1, b), *) = (\mathbb Z
\times \mathbb Z, +) \)