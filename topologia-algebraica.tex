\documentclass[letterpaper]{article}

\usepackage[letterpaper]{geometry}
\usepackage[log-declarations=false]{xparse}
\usepackage[quiet]{fontspec}
\usepackage{amsmath}
\usepackage{amsthm}
\usepackage{amsfonts}
\usepackage{enumerate}
\usepackage{polyglossia}
\usepackage{tikz}
\usetikzlibrary{arrows,shapes,trees}

\setdefaultlanguage{spanish}

\theoremstyle{definition}
\newtheorem{definicion}{Definicion}

\theoremstyle{plain}
\newtheorem{teorema}{Teorema}

\begin{document}
\title{Topologia algebraica}
\author{Ruben Astudillo \\ 201021009-k}
\date{\today}
\maketitle

\section{Prefacio}
Topologia nos da un idioma para entender diferentes espacios a traves de
propiedades como compacidad, cerrado/abierto y nociones de separabilidad.
Siempre estudiando estos en visiones particulares, sin hacer enfasis
en \emph{clasificar} diferentes espacios con respecto a sus propiedade invariantes
topologias. Este deseo proviene del mundo de el algebra, en particular de
teoria de grupos que habia logrado en el siglo XIX clasificar
progresivamente grupos de diferentes ordenes modulo isomofismo.
Resulta ser que el intento ``simple'' de clasificar espacios
en base a nociones puramente topologicas resulta dificil pues por ejemplo
nociones como compacidad no ayudan a distinguir fundamentalmente las
diferencias entre por ejemplo \(\mathbb{R}^2\) y  \(\mathbb{R}^3\) pues las
dimensiones son transparentes para muchos conceptos (siempre que sean finitos).

Si en vez de tomar compacidad se toma la nocion de arco-conexidad,
vemos que esta si respeta la dimension de los espacios, ya que los
caminos estan coartados por el tipo agujeros del espacio; los cuales
pueden tener distintas dimensiones. Resulta que las clases de
equivalencia de las curvas con respecto a los distintos agujeros de los
espacios muestra estructura de grupo, cuyas clasificaciones ya son
estudiadas. De manera que podemos reutilizar el trabajo en invariantes
algebraicas para entender nociones topologicas. En este trabajo nos
enfocaremos en los grupos de espacios con respecto a agujeros simples
(aquellos isomorfos a espacios uni-dimensionales) solo por la
familiaridad. Pero todo el trabajo puede realizarse tambien para agujeros
de dimensiones superiores.

\section{Grupo fundamental}

\subsection{Arcos y homotopias}
Nuestros bloques base de construccion seran los arcos y las
deformaciones continuas entre ellos llamadas \emph{homotopias}.

\begin{definicion}[arco]
  Un arco en el espacio \(X\) es una funcion continua \(f : I \to X \)
  donde \(I = [0,1]\)
\end{definicion}
La definicion tradicional es ligeramente mas general cambiando que \(I\)
sea un espacio de 1 dimension. Pero siempre podemos recuperar esta
rescalando la distancia.

\begin{definicion}[homotopia]
  Dados dos arcos \(f,g : I \to X\), diremos que \(f\) es homotopico a
  \(g\) si existe una funcion continua \(F : I \times I \to X \) tal que
  \[ \begin{matrix}
      F (x, 0) = f(x) & F (x, 1) = g(x)
     \end{matrix}
  \]
  Donde \(F\) sera llamada una homotopia entre \(f\) y \(g\). La
  existencia de dicha relacion entre dos funciones sera denotada por \(f
  \simeq_{h} g\).
\end{definicion}
% todo(slack): hace curva girada en 90 grados y lineas. fijate que no sea
% una arco homotopia
% \begin{tikzpicture}
%   \draw (2,0) to [out=0, in=-90] (2.5, 0.5) to [out=90, in=0] (2,1)
%     to [out=180, in=-90] (1.5, 1.5) to [out=90, in=180] (2,2);
% \end{tikzpicture}
Podemos pensar en el segundo argumento de una homotopia como el grado de
deformación entre dos funciones.
Si tratamos con arcos \(f,g : I \to X\) que posean los mismos puntos
iniciales y finales, es decir
\[ \begin{matrix}
    f(0) = g(0) = x_0  & & f(1) = g(1) = x_1
   \end{matrix}
\]
podemos definir una relacion homotopica ligeramente mas fuerte
\begin{definicion}[arco homotopia]
  \(f,g : I \to X\) son \textbf{arco homotopicas} si tienen los mismos
  puntos inicial \(x_0\) y final \(x_1\) y existe una homotopia entre
  ellos tales que cumpla
  \[
    \begin{matrix}
      F(s,0) = f(s) & F(s,1) = g(s) \\
      F(0,t) = x_0  & F(1,t) = x_1
    \end{matrix}
  \]
  La existencia de esta relacion entre dos funciones sera denotada por
  \(f \simeq_{ah} g\)
\end{definicion}
% todo(slack) hacer algun dibujo de las homotopias entre las curvas
Una vez definida esta relaciones, una pregunta natural es si estas definen
una relacion de equivalencia, ya que nos gustaria identificar clases de
curvas como elementos de alguna teoria algebraica. La respuesta es
afirmativa para ambas relaciones (el punto de inicio y final no juegan
papel), se mostrara para las homotopias.

\begin{teorema}
  \(\simeq_h\) es una relacion de equivalencia
\end{teorema}
\begin{proof}
  Hemos de probar que esta relacion cumple la reflexividad, simetria y
  transitividad. Sean \(f,g,h : I \to X\) tres arcos arbitrarios.

  La reflexividad es directa pues la funcion \(F(x,t) = f(x)\) es una
  deformacion continua de \(f\) a \(f\). Por tanto \(f \simeq_h f\)

  La simetria se obtiene de invertir el grado de deformacion de la
  homotopia original. Formalmente, dado \(f \simeq_h g\) tenemos una
  homotopia entre estas \((x,t) \mapsto F(x,t)\). A partir de aqui
  podemos definir
  \[ (x,t) \mapsto \hat{F}(x,t) = F(x,1-t)\]
  con \(\hat{F}\) una homotopia entre \(g\) y \(f\). Por tanto \(g
  \simeq_h f\).

  La transitividad se obtiene a partir de la idea de dividir \(I\) en dos
  intervalos donde se deformen individualmente cada homotopia.
  Formalmente dado \(f \simeq_h g\) y \(g \simeq_h h\) representas por
  las homotopias \(F\) y \(G\) respectivamente, se define
  \[ FG(x,t) = \begin{cases}
                 F(x,2t) & t \in [0,\frac{1}{2}] \\
                 G(x,2t - 1) & t \in [ \frac{1}{2} , 1]
               \end{cases}
  \]
  Esta es una deformacion continua claramente en \((x,t) \in I \times [0,
  \frac{1}{2}) \cup I \times (\frac{1}{2}, 1]\). La continuidad en \(I
  \times \{\frac{1}{2}\}\) proviene de la consistencia en dicho punto de
  ambas homotopias
  \[ F(x,2 \cdot \frac{1}{2}) = g(x) = G(x, 2 \cdot \frac{1}{2} - 1)\]
  lo que nos permite utilizar el lema del pegamiento para obtener la
  continuidad de \(FG\). Obteniendo asi \(f \simeq_h h\).
\end{proof}

Existe una homotopia clasica que es muy util como para no conocer. Dadas
\(f,g : I \to X\), si existe un espacio convexo que contenga el recorrido
de \(f\) y \(g\), se puede definir la homotopia de lineal recta entre
estas
\[ (x,t) \mapsto F(x,t) = (1-t) \cdot f(x) + t \cdot g(x) \]
la cual es claramente continua pues es combinacion lineal de funciones continuas.
%todo(slack): otro ejemplo de herramienta de construccion de homotopias

Con estas definiciones ya podemos empezar a hablar de \([f],[g]\) clases
de equivalencia de funciones bajo una relacion homotopica, aunque nuestro
interes principal siempre sera en clases de funciones \emph{arco
homotopicas}.

Para dos arcos \(f,g : I \to X\) que compartan punto final e inicial
(\(f(1) = g(0)\)), podemos definir una nocion de producto que corresponde
a seguir el camino de \(f\) y luego de \(g\). En formulas se define \(f *
g\) cuando \(f(1) = g(0)\) como
\[ (f*g) (x) = \begin{cases}
    f(2x) & x \in [0,\frac{1}{2}] \\
    g(2x - 1) & x \in [\frac{1}{2} , 1]
  \end{cases}
\]
la cual sigue siendo una funcion continua en virtud del lema del
pegamiento. Esta construccion se puede reutilizar para clases
\emph{arco}-homotopicas \([f],[g]\) que compartan punto final e inicial
respectivamente para definir un producto de clases de equivalencia
\[ [f] * [g] := [f * g]\]
El cual esta bien definido pues si \(f \simeq_{ah} f'\) a traves de \(F\)
y \(g \simeq_{ah} g'\) a traves de \(G\)
\[H(s,t) = \begin{cases}
    F(2x,t) & x \in [0, \frac{1}{2}] \\
    G(2x - 1, t) & x \in [\frac{1}{2} , 1]
  \end{cases}
\]
Es la homotopia que relaciona a los elementos de \([f*g]\) la cual es
continua en virtud otra vez del lema del pegamiento.
\section{Cubrimientos pares}

\section{Van Kampen}

\section{Nociones categoricas}

\end{document}
