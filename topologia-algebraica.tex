\documentclass[letterpaper]{article}

\usepackage[letterpaper]{geometry}
\usepackage[log-declarations=false]{xparse}
\usepackage[quiet]{fontspec}
\usepackage{amsmath}
\usepackage{amsfonts}
\usepackage{enumerate}
\usepackage{polyglossia}
\usepackage{tikz}
\usetikzlibrary{arrows,shapes,trees}

\setdefaultlanguage{spanish}

\begin{document}
\title{Topologia algebraica}
\author{Ruben Astudillo \\ 201021009-k}
\date{\today}
\maketitle

\section{Prefacio}
Topologia nos da un idioma para entender diferentes espacios a traves de
propiedades como compacidad, cerrado/abierto y nociones de separabilidad.
Siempre estudiando estos en visiones particulares, sin hacer enfasis
en \emph{clasificar} diferentes espacios con respecto a sus propiedade invariantes
topologias. Este deseo proviene del mundo de el algebra, en particular de
teoria de grupos que habia logrado en el siglo XIX clasificar
progresivamente grupos de diferentes ordenes modulo isomofismo.
Resulta ser que el intento ``simple'' de clasificar espacios
en base a nociones puramente topologicos resulta dificil pues por ejemplo
nociones como compacidad no ayudan a distinguir fundamentalmente las
diferencias entre por ejemplo \(\mathbb{R}^2\) y  \(\mathbb{R}^3\) pues las
dimensiones son transparentes para muchos conceptos (siempre que sean finitos).

Si en vez de tomar compacidad se toma la nocion de arco-conexidad,
vemos que esta si respeta la dimension de los espacios, ya que los
caminos estan coartados por el tipo agujeros del espacio; los cuales
pueden tener distintas dimensiones. Resulta que las clases de
equivalencia de las curvas con respecto a los distintos agujeros de los
espacios muestra estructura de grupo, cuyas clasificaciones ya son
estudiadas. De manera que podemos reutilizar el trabajo en invariantes
algebraicas para entender nociones topologicas. En este trabajo nos
enfocaremos en los grupos de espacios con respecto a agujeros simples
(aquellos isomorfos a espacios uni-dimensionales) solo por la
familiaridad. Pero todo el trabajo puede realizarse tambien para agujeros
de dimensiones superiores.

\end{document}
