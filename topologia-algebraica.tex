\documentclass[letterpaper]{article}

\usepackage[letterpaper]{geometry}
\usepackage[log-declarations=false]{xparse}
\usepackage[quiet]{fontspec}
\usepackage{amsmath}
\usepackage{amsthm}
\usepackage{amsfonts}
\usepackage{enumerate}
\usepackage{polyglossia}
\usepackage{tikz}
\usetikzlibrary{arrows,shapes,trees}

\setdefaultlanguage{spanish}

\theoremstyle{definition}
\newtheorem{definicion}{Definicion}

\theoremstyle{plain}
\newtheorem{teorema}{Teorema}

\theoremstyle{plain}
\newtheorem{lema}{Lema}

\theoremstyle{remark}
\newtheorem{anotacion}{Anotacion}

\begin{document}
\title{Topologia algebraica}
\author{Ruben Astudillo \\ 201021009-k}
\date{\today}
\maketitle

\section{Prefacio}
Topología nos da un idioma para entender propiedades de diferentes
espacios a través de nociones como compacidad, cerrado/abierto y
diferentes tipos
de separabilidad. Gran parte de estas propiedades tienen un carácter
local, es decir no son propiedades que definan al espacio total si no mas
bien por subparte. En el siglo XIX el mundo del álgebra, en
particular la teoría de grupos había intentado y logrado clasificar
progresivamente grupos de diferentes ordenes modulo isomorfismo. Habían
logrado esto a través de llamadas invariantes algebraicas, las cuales
eran preservadas a través de los isomorfismo de grupos. En topología se
deseaba hacer un tratamiento parecido, por lo cual necesitaban una
invariante topológica que nos permitiera clasificar espacios. La elección
de esta invariante no necesariamente nos da teorías útiles, pues puede
ser muy general y no distinguir en propiedades que se consideran
importantes. Uno puede tomar como ejemplo que si nuestra propiedad es
la compacidad, esta no hace un buen papel en distinguir entre
\(\mathbb{R}^2\) y \(\mathbb{R}^3\).

Si hemos de tomar una propiedad como invariante topológica sobre la cual
construir una teoría, esta no debe ser local. Idealmente nos gustaría
construir sobre alguna propiedad del espacio que fuera característica de
todo el espacio, algo en general notoriamente difícil. Resulta que si nos
reducimos a espacios arco-conexos podemos construir una respuesta a través
de los grupos fundamentales. Esta característica es construida sobre
clases de equivalencias de caminos cerrados quienes puedan deformarse
entre si y permiten reificar información de los ``agujeros'' del espacio.
Resulta que la elección de esta respuesta tiene la consecuencia feliz de
tener estructura de grupo subyacente, por lo que nos permite reutilizar
el trabajo ya realizado en el mundo del álgebra para nuestro beneficio.
Mas aun, la construcción en si da cabida a interpretaciones de
teoría de categorías, por lo que sirve de buena introducción a un
lenguaje mas general.

\section{Grupo fundamental}

\subsection{Arcos y homotopias}
Nuestros bloques base de construccion seran los arcos y las
deformaciones continuas entre ellos llamadas \emph{homotopias}.

\begin{definicion}[arco]
  Un arco en el espacio \(X\) es una funcion continua \(f : I \to X \)
  donde \(I = [0,1]\)
\end{definicion}
La definicion tradicional es ligeramente mas general cambiando que \(I\)
sea un espacio de 1 dimension. Pero siempre podemos recuperar esta
rescalando la distancia.

\begin{definicion}[homotopia]
  Dados dos arcos \(f,g : I \to X\), diremos que \(f\) es homotopico a
  \(g\) si existe una funcion continua \(F : I \times I \to X \) tal que
  \[ \begin{matrix}
      F (x, 0) = f(x) & F (x, 1) = g(x)
     \end{matrix}
  \]
  Donde \(F\) sera llamada una homotopia entre \(f\) y \(g\). La
  existencia de dicha relacion entre dos funciones sera denotada por \(f
  \simeq_{h} g\).
\end{definicion}
% todo(slack): hace curva girada en 90 grados y lineas. fijate que no sea
% una arco homotopia
% \begin{tikzpicture}
%   \draw (2,0) to [out=0, in=-90] (2.5, 0.5) to [out=90, in=0] (2,1)
%     to [out=180, in=-90] (1.5, 1.5) to [out=90, in=180] (2,2);
% \end{tikzpicture}
Podemos pensar en el segundo argumento de una homotopia como el grado de
deformación entre dos funciones.
Si tratamos con arcos \(f,g : I \to X\) que posean los mismos puntos
iniciales y finales, es decir
\[ \begin{matrix}
    f(0) = g(0) = x_0  & & f(1) = g(1) = x_1
   \end{matrix}
\]
podemos definir una relacion homotopica ligeramente mas fuerte
\begin{definicion}[arco homotopia]
  \(f,g : I \to X\) son \textbf{arco homotopicas} entre si, si tienen los mismos
  puntos inicial y final \(x_0, x_1\) respectivamente y existe una homotopia entre
  ellos tales que cumpla
  \[
    \begin{matrix}
      F(s,0) = f(s) & F(s,1) = g(s) \\
      F(0,t) = x_0  & F(1,t) = x_1
    \end{matrix}
  \]
  La existencia de esta relacion entre dos funciones sera denotada por
  \(f \simeq_{ah} g\)
\end{definicion}
% todo(slack) hacer algun dibujo de las homotopias entre las curvas
Una vez definida esta relaciones, una pregunta natural es si estas definen
una relacion de equivalencia, ya que nos gustaria identificar clases de
curvas como elementos de alguna teoria algebraica. La respuesta es
afirmativa para ambas relaciones (el punto de inicio y final no juegan
papel), se mostrara para las homotopias.

\begin{teorema}
  \(\simeq_h\) es una relacion de equivalencia
\end{teorema}
\begin{proof}
  Hemos de probar que esta relacion cumple la reflexividad, simetria y
  transitividad. Sean \(f,g,h : I \to X\) tres arcos arbitrarios.

  La reflexividad es directa pues la funcion \(F(x,t) = f(x)\) es una
  deformacion continua de \(f\) a \(f\). Por tanto \(f \simeq_h f\).

  La simetria se obtiene de invertir el grado de deformacion de la
  homotopia original. Formalmente, dado \(f \simeq_h g\) tenemos una
  homotopia entre estas \((x,t) \mapsto F(x,t)\). A partir de aqui
  podemos definir
  \begin{equation}
    \label{eq:homotopy-simetry}
    (x,t) \mapsto \hat{F}(x,t) = F(x,1-t)
  \end{equation}
  con \(\hat{F}\) una homotopia entre \(g\) y \(f\). Por tanto \(g
  \simeq_h f\).

  La transitividad se obtiene a partir de de dividir \(I\) en dos
  intervalos donde se deformen individualmente cada homotopia al doble
  del grado. Formalmente dado \(f \simeq_h g\) y \(g \simeq_h h\)
  representadas por las homotopias \(F\) y \(G\) respectivamente, se define
  \[ FG(x,t) = \begin{cases}
                 F(x,2t) & t \in [0,\frac{1}{2}] \\
                 G(x,2t - 1) & t \in [ \frac{1}{2} , 1]
               \end{cases}
  \]
  Esta es una deformacion continua claramente en \((x,t) \in I \times [0,
  \frac{1}{2}) \cup I \times (\frac{1}{2}, 1]\). La continuidad en \(I
  \times \{\frac{1}{2}\}\) proviene de la consistencia en dicho punto de
  ambas homotopias
  \[ F(x,2 \cdot \frac{1}{2}) = g(x) = G(x, 2 \cdot \frac{1}{2} - 1)\]
  lo que nos permite utilizar el lema del pegamiento para obtener la
  continuidad de \(FG\). Obteniendo asi \(f \simeq_h h\).
\end{proof}

Un ejemplo clasico de una homotopia es la homotopia de linea recta. Dadas
\(f,g : I \to X\), si existe un espacio convexo que contenga el recorrido
de \(f\) y \(g\), esta homotopia se define por
\[ (x,t) \mapsto F(x,t) = (1-t) \cdot f(x) + t \cdot g(x) \]
la cual es claramente continua pues es combinacion lineal de funciones continuas.
%todo(slack): otro ejemplo de herramienta de construccion de homotopias

Con estas definiciones ya podemos empezar a hablar de \([f]\) clases
de equivalencia de funciones bajo una relacion homotopica
\[ [f] = \{ g : I \to X \mid f \simeq_{ah} g \} \]
Notar que la construccion es valida para las dos relaciones aunque por
nuestra meta de construir el grupo fundamental nos interesa
principalmente la relacion arco homotopica (tema a ser tratado mas adelante).

\subsection{Grupoide de arcos}
Si tenemos dos arcos continuos \(f,g\) tales que el punto final de \(f\)
sea el punto inicial de \(g\) nos da la idea de que podemos construir un
camino continuo que recorra \(f\) luego recorriendo \(g\). Esta idea es
conocida formalmente como el producto de dos arcos.

\begin{definicion}[Producto de arcos]
Para dos arcos \(f,g : I \to X\) tales que
\(f(1) = g(0)\), se define el producto \(f * g \)
\[ (f*g) (s) = \begin{cases}
    f(2s) & s \in [0,\frac{1}{2}] \\
    g(2s - 1) & s \in [\frac{1}{2} , 1]
  \end{cases}
\]
la cual sigue siendo una funcion continua en virtud del lema del
pegamiento.
\end{definicion}

Esta construccion se puede reutilizar para clases
\emph{arco}-homotopicas \([f],[g]\) que compartan punto final e inicial
respectivamente para definir un producto de clases de equivalencia
\[ [f] * [g] := [f * g]\]
El cual esta bien definido pues si \(f \simeq_{ah} f'\), \(g \simeq_{ah}
g'\) a traves de \(F, G\) respectivamente
\[H(s,t) = \begin{cases}
    F(2s,t) & s \in [0, \frac{1}{2}] \\
    G(2s - 1, t) & s \in [\frac{1}{2} , 1]
  \end{cases}
\]
Es la homotopia que relaciona a los arcos que sean homotopicos a
\(f*g\). Esta ademas es continua en virtud otra vez del lema del pegamiento.

Con esta operacion binaria, una pregunta a hacerse es si tiene
\((X/\simeq_{ah} , *)\) posee
estructura de grupoide. Esto equivale a cumplir 3 propiedades
\begin{enumerate}
\item \textbf{Asociatividad} Si \([f] * ([g] * [h])\) esta definido entonces
  tambien lo debe estar \(([f] * [g]) * [h]\) y deben de ser iguales.
\item \textbf{Identidades izquierda y derecha} Para todo \([f]\) con
  \(x_0, x_1\) puntos inicial y final respectivamente, debe de
existir elementos \([k_{0}], [k_{1}]\) tales que
\[ \begin{matrix}
    [f] * [k_{1}] = [f] & & [k_{0}] * [f] = [f]
  \end{matrix}
\]
\item \textbf{Inverso} Para todo \([f]\) clase de arcos con \(x_0, x_1\)
  puntos inicial y final respectivamente debe de existir un elemento
  \([\bar{f}]\) que cumpla
\[ \begin{matrix}
    [f] * [\bar{f}] = [k_{x_0}] & & [\bar{f}] * [f] = [k_{x_1}]
  \end{matrix}
\]
\end{enumerate}
Para probar esto necesitamos primero definir a nuestros candidatos de
\(k_{x_1}, k_{x_2}, \bar{f}\) ademas de algunas funciones auxiliares,
iniciando por los mapeos constantes e identidad en \(I\)
\[ \begin{matrix}
     e_0 : & I \to I & e_0(t) := 0 \\
     e_1 : & I \to I & e_1(t) := 1 \\
     i :   & I \to I & i(t) := t \\
     \bar{i} : & I \to I & i(t) := 1 - t
   \end{matrix}
   \]
Para todo arco \(f : I \to X \), el elemento \(\bar{f} : I \to X \) esta
definido (en el espiritu de \eqref{eq:homotopy-simetry}) por
\[ \bar{f} (s) := f (1 - s) \]
Para todo \(x \in X \) se define la curva constante
\begin{align*}
  k_x : &I \to X \\
        &t \mapsto x
\end{align*}
Con esto podemos afrontar la demostración
\begin{proof}
Se procedera \(2 \to 3 \to 1\) con lemas en medio. Sea \(f : I \to X\) el
representante de \([f]\). sean \(x_0, x_1\) los puntos inicial y final de
\(f\) respectivamente.

\paragraph{(2).} Dados \(i,e_1\) arcos en \(I\) definidos anteriormente,
es claro que \(i * e_1\) es tambien un arco (continuo) en \(I\); mas aun,
ya que \(I\) es convexo se tiene \(i * e_1 \simeq_{ah} i\) con la
homotopia de la linea recta entre estos. Se desprende de esto ultimo que
\( [i] = [i * e_1]\). Para continuar necesitamos el siguiente lema
\begin{lema}[Distributividad de la composicion sobre producto.]
\label{lema:dist-composicion-producto}
\[\forall a,b : I \to I,\ f \circ (a * b) = (f \circ a) * (f \circ b) \]
\end{lema}
\begin{proof}
  \[ f \circ (a*b) (s) =
    \begin{cases}
      f (a(2s)) & s \in [0,\frac{1}{2}] \\
      f (b(2s - 1)) & s \in [\frac{1}{2} , 1]
    \end{cases}
    = (f \circ a) * (f \circ b)
  \]
\end{proof}
Recordemos ademas que la composición de funciones continuas es continua.
Luego esto nos permite afirmar en especifico que
\[ f \circ (i * e_1) \simeq_{ah} (f \circ i) * (f \circ e_1) \] en virtud
de la composición de \(f\) con la homotopia de linea recta entre \(i *
e_1 \) y \(i\). Obteniendo asi que
\[ [f \circ (i * e_1)] = [(f \circ i) * (f \circ e_1)] \] pero al reducir
las composición obtenemos
\[
  \begin{matrix}
    f \circ i = f & f \circ e_1 = k_{x_1} & f \circ (i * e_1) = f *
    k_{x_{1}} \\
  \end{matrix}
\]
\[ \implies [f] * [k_{x_1}] := [f * k_{x_1}] = [f \circ (i * e_1)] = [ f
  \circ i] = [f] \]
probando (2).

\paragraph{(3).} De manera similar a la pregunta anterior, notar que \(i,
\bar{i}\) definidos anteriormente cumplen
\[ i * \bar{i} \simeq_{ah} e_0 \]
por la homotopia de linea recta en el espacio \(I\) que es convexo.
Utilizando el lemma \eqref{lema:dist-composicion-producto} tenemos que
\[ f * \bar{f} = f \circ (i * \bar{i}) \simeq_{ah} f \circ e_0 =
  k_{x_0} \]
\[ \implies [f] * [\bar{f}] = [k_{x_0}] \]
De manera analoga se prueba la inversa por la izquierda.

\paragraph{(1).}
% todo(slack)
\end{proof}

\subsection{Grupo fundamental}
Sobre nuestros arcos tenemos una estructura de grupoide porque no siempre
podemos hacer coincidir los puntos inicial y final. Si nos fijaramos en
un solo punto \(x_0 \in X\) tal que todas los arcos \(f : I \to X\)
cumplieran que
\[ f(0) = x_0 = f (1) \]
Entonces las propiedades que demostramos anteriormente nos dirian que el
espacio \((X / \simeq_{ah}, *)\) es un grupo. Una pregunta natural ha
hacerse es si la eleccion de punto influye en el tipo de grupo que se
obtiene en \((X / \simeq_{ah}, *)\). En general la respuesta es
afirmativa, lo cual complica nuestra tarea puesto que queremos definir
una invariante topologica que dependa del espacio total. Pero podemos
recuperar nuestro objetivo si nos restringimos a espacios \(X\) que sean
arco-conexos.
\begin{definicion}[arco-conexidad]
  Diremos que el espacio \(X\) es \emph{arco conexo} si para todo par de
  puntos \(x,y \in X\) existe un arco continuo \(f : I \to X\) tal que
  cumpla
  \[ f(0) = x \qquad f(1) = y \]
\end{definicion}
Para estos espacios, denotamos a \((X / \simeq_{ah}, *)\) como
\(\pi_1(X,x_0)\) y le llamaremos \textbf{grupo fundamental} de \(X\) con
respecto a \(x_0\). Es claro que la importancia del punto \(x_0\) es poca
considerando la arco-conexidad. Puesto que si tengo \([f] \in
\pi_1(X,x_0)\) y \(\alpha : I \to X\) un arco entre \(x_0\) y \(x_1\),
existe un camino \([\hat{\alpha} * f * \alpha] \in \pi_1(X,x_1)\), es
decir un isomorfismo.
% La razon de porque aun guardamos a \(x_0\) como
% informacion del grupo es que a priori los isomorfismos entre \(\pi_1 (X,
% x_0)\) y \(\pi_1 (X, x_1)\) pueden depender del arco \(\alpha\) elegido,
% de los cuales ningun

\section{Cubrimientos pares}

\section{Van Kampen}

\section{Nociones categoricas}
encontre una construccion de suspensiones topologicas, al parecer me
sirver para hacer una adjuncion entre estos los espacios loop
\end{document}
