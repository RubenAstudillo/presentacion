\documentclass[letterpaper]{article}

\usepackage[letterpaper]{geometry}
\usepackage[log-declarations=false]{xparse}
\usepackage[quiet]{fontspec}
\usepackage{amsmath}
\usepackage{amssymb}
\usepackage{mathtools}
\usepackage{amsfonts}
\usepackage{enumerate}
\usepackage{polyglossia}
\usepackage{unicode-math}

\newcommand{\vank}{\emph{Seifert-van Kampen} }

\setdefaultlanguage{spanish}

\begin{document}
\title{Discusión}
%Esto ira acompañando a la presentacion. Cada secciones sera una nueva
%pagina de la presentacion.

\subsection*{Pagina 2}
La motivación principal es definir una invariante topológica que
denominaremos grupos fundamental de un espacio topológico \(X\). Esta
tendrá una estructura algebraica (de grupo) y podrá reutilizar mucho del
trabajo de clasificación de grupos para clasificar espacio topológicos.

\paragraph{pausa 2}
La definición de esta invariante ocurre al estudiar los caminos
admisibles en un espacio \(X\) y las deformaciones que sean admisibles
entre estos dado el espacio \(X\).

\subsection*{Pagina 3}
Solo para afirmar el lenguaje, un camino sobre un espacio topológico
\(X\) es una función continua
\[ f : [0,1] \to X \]

\paragraph{}
Una homotopía es la formalización de las deformaciones admisible entre
dos funciones. Dicho formalmente \texttt{Mostrar definición en
  diapositiva} es dicha función continua donde el primer parámetro
evalúa en la curvas y el segundo corresponde a un grado de deformación

\subsection*{Pagina 4}
Si dos caminos son compatibles (ie tienen el mismo punto final e
inicial), entonces se puede definir el producto de ellos de la siguiente
forma \texttt{mostrar le definición de producto} \\

Es de mayor interés los caminos cerrados (con mismo punto final e
inicial) pues el operación anterior esta bien definida siempre.

Ademas tenemos para todo camino podemos definir su inversa moviendo el
parametro en sentido inverso en \(I\).

\subsection*{Pagina 5}
Para una homotopía se puede hacer un proceso análogo. Se utiliza la
misma notación.

\subsection*{Pagina 6}
Con esto definimos el siguiente conjunto
\texttt{mostrar \(\pi\)} el cual junto a la operación de producto
\texttt{mostrar \((*)\)} forman el grupo fundamental de \(X,x_0\). Notar
que esta operación sobre clases de equivalencias esta bien definida por
la definición de producto de homotopias.

\subsection*{Pagina 7}
Si queremos clasificar espacios topológicos, que el punto de origen
afecte al grupo fundamental es insatisfactorio. Esto lo pedimos ver en
el ejemplo de \(D^2\) union con \(S^1\) con diferentes origines, un
espacio no conexo. Si tomamos \(x_0\) en \(D^2\) de un grupo fundamnetal
trivial mientras que si lo tomamo para \(S^1\) este no es trivial. Por
esto es que en general esta invariante se usa principalmente para
espacio arco-conexos. De manera que siempre estamos a una conjugación de
grupo de cualquier otro punto base.

\paragraph{Pausa}
Aun en espacio arco-conexos vale la pena mantener la información del
punto de origen, pues existe una caracterización de grupo fundamental
Abeliano enumerando los posibles isomorfismos de grupo del grupo
fundamental con distintos puntos bases.

\subsection*{Pagina 8}
Un ejemplo intuitivo seria estudiar el espacio \(R^2\) menos un punto.
En este caso tomaremos \((1,1)\) pero puede ser cualquier otro. Se nos
ocurren tres grandes clases de caminos
\begin{itemize}
\item Aquellos que son homomorficos a \(k_0\). Estos son aquellos que no
  encierran a \((1,1)\) pues estan contenidos en un subespacio convexo y
  podemos utilizar la siguiente homotopia
  \[ L (t,\lambda) = \lambda f (t) - (1 - \lambda) k_{x_0} (t)\]
\item Aquellos que encierren a \((1,1)\) una vez. Los cuales pertenecen
  a su propia clase que denotaremos \(\mathbb F ^1\)
\item Aquellos que encienrren a \((1,1)\) mas veces. Siempre se pueden
  ver como homomorficos a \(\mathbb F * \dots * \mathbb F \)
\end{itemize}
Notamos el producto entre estas distintas clases equivale a levantar la
pontencia de la clase \(\mathbb F\), luego este grupo es isomorfo al
aditivo \((\mathbb Z , +)\). Esto es solo un ejemplo intuitivo,
necesitamos herramientas formales para determinar grupos fundamentales

\subsection*{Pagina 9}
Es acaso el grupo fundamental una invariante? Esto equivale a mostrar
que si tenemos un homeomorfismo entre espacios, tambien tenemos un
isomorfismo entre grupos

\paragraph{Pausa 1}
Esto es cierto utilizando la siguiente construccion \texttt{mostrar}.
Este es un homorfismo de grupos pues por defininicion \(f_*\) se
distribuye sobre productos \(*\). La biyectividad es consecuencia de la
biyectividad de \(f\).

\subsection*{Pagina 10}
Esta construccion es general, esto es un functor entre categorias
\(\mathscr{Top}_* \to \mathscr{Grp}\). Lo que hace el functor es refleja
la estructura de la categoria de partida en la categoria de llegada
preservando la relacion dada por los operadores composicion respectivos.

Se nota que uno puede pensar en una categoria como un grafo
(posiblemente disconexo) con morfismos entre los objetos, donde existe
una composicion de morfismos y para todo objetos tiene un unico morfismo
identidada si mismo.

\subsection*{Pagina 11}
Ahora nuestra meta sera obtener formas de calcular explicitamente el
grupo fundamental. En general esto es dificil pero hay algunos enfoques
que nos dan respuestas en gran cantidad de casos. Estos son (decir)

\subsection*{Pagina 12}
Un par de equivalencias homotopicas son dos funcinoes entre espacios que
cumplan la relacion de la definicion. La presencia de un par de ellas
nos dice que estos espacios tiene mismo grupo fundamental. ¿como
probamos esto? Necesitamos algunos lemas tecnicos.

\subsection*{Pagina 13}
El primer lema es trivial, uno lo puede probar por contradiccion. El
segundo nos relaciona funciones que estan homotopicamente relacionadas
con una igualdad en sus levantamientos respectivos. Con estos en mente
procedemos a la demostracion del teorema anterior.

\subsection*{Pagina 14}
\texttt{Notar que esto esta todo en la diapositiva, puededes mostrarlo directamente}.
Notemos primero que
\[ g \circ f \simeq Id : X \to X \implies g_* \circ f_{*,x_0} = \hat
  \alpha \circ Id_* \]
lo cual implica que \(g_*\) es sobreyectivo. Analogamente se tiene que
\[ f \circ g \simeq Id : Y \to Y \implies f_{*,x_1} \circ g_* = \hat
  \beta \circ Id_* \]
Lo que implica que \(g_*\) es inyectivo. Luego \(g_*\) es biyectivo y
por tanto en la ecuacion anterior, se puede despejar \(f_{*,x_0}\)
\[ f_{*,x_0} = \hat \alpha \circ \left( g_{*} \right)^{-1}\]
por lo que dado que es una composicion de funciones biyectivas,
\(f_{x_0,*}\) es biyectivo ademas de ser un homomorfismo de grupos.

\subsection*{Pagina 15}
Estos tres espacios no son homemorfos. El primero con el segundo son dos
bolas unidas por una linea y el segundo son dos bolas unidas por un
punto. Si fueran homemorfos, se tendria un homeomofismo entre una linea
y un punto. Analogamente para la tercera figura

Sin embargo estas son del mismo tipo homotopico pues todas son
homotopicamente equivalentes al disco \(D^2\) menos dos puntos. De hecho
las equivalencias homotopicas son las inyecciones y son homotopicas a la
identidas por el siguiente diagrama \texttt{ver diagrama}.

\subsection*{Pagina 16}
La banda de mobius se presenta como el cuadrado \(I^2\) con los lados
identificados en sentido inverso. Tomamo la recta vertical media y
notamos que siempre podemos hacer una homotipa a esta. Luego notamos que
esta recta (con los extremos identificados) es homemomorfa a \(S^1\).
Luego la banda de mobius tiene el mismo grupo fundamental de \(S^1\)

\subsection*{Pagina 17}
Con la nocion de equivalencias homotopicas podemos enfocarnos en las
familia de funciones continuas homotopicamente equivalentes que entre
dos espacios topologicos. Esto nos lleva a definir la categoria HoTop
que formaliza nuestros resultados \texttt{mostrar definicion en
  diapositiva}

\subsection*{Pagina 18}
En particular podemos ver a \(Top_*\) como una subcategoria mediante un
functor inyeccion adecuado. Notar que todo par de espacios topologicos
homeomorfos tienen un par de equivalencias en HoTop.

Se extiende el functor \(\pi\) a trabajar desde \(HoTop\). Todo par de
espacios homotopicos que tengan un par de equivalencias homotopicas,
cualquiera de estas equivalencias al ser levantas de \(f_*\) mediante
\(\pi\) se convierte en un isomorfismo de grupos.

\subsection*{Pagina 19}
Se tiene productos en esta categoria y el functor \(\pi\) preserva esta
estructura. Esto es por la caracterizacion de como son las funciones
continuas en un espacio producto.

\subsection*{Pagina 20}
Tambien se tiene un co-producto dado por la siguiente construccion
\texttt{ver definicion}. La pregunta es si el functor preserva esta
estructura ? La respuesta es afirmativa, se vera mas adelante con \vank

Por ahora estudiaremos otros resultados que pueden ser resumidos en
terminos categoricos. Primero partiendo por cubrimientos y luego por el
teorema de \vank

\subsection*{Pagina 21}
Un cubrimiento es la siguiente definicion \texttt{ver defincion}. Se
puede probar que todo cubrimiento es un mapeo abierto, asi que es es una
funcion con propiedades bastante fuertes.

Tambien tenemos la fibra de un punto, que tendra relacion con el
cubrimiento universal mas adelantante

\subsection*{Pagina 22}
Un cubrimiento es interesante porque nos permite estudiar las curvas de
\(X\) en un espacio diferencia (idealmente mas simple) pues las curvas
de \(X\) se pueden levantar a \(\tilde X\). Lo teoremas que nos permiten
hacer esto son dos:

El levantamiento de curvas nos dice que para toda curva cerrada en
\(x_0\) existe un levantamiento en \(\tilde X\) que comienza en en
\(\tilde x_0\) donde este ultimo lo elegimos nosotros. No hay
requerimiento de que este termine en \(\tilde x_0\) (es decir se puede
desenrrollar). La eleccion de \(\tilde x_0\) es nuestra mientras
pertenesca a la fibra \(p^{-1} (x_0)\). Mas aun, una vez fijado este
punto de inicio en \(\tilde X\), este levantamiento es unico.

Existe tambien un teorema de levantamiento de homotopias con las mismas
semanticas.

\subsection*{Pagina 23}
La idea de la demostracion es la siguiente \texttt{mostrar esquema}

\subsection*{Pagina 24}
Cuando se tiene dividido ya el intervalo \(I\), tomamos el conjunto que
es cubierto por \(p\) en \(x_0\). Este tiene una familia de conjuntos
disconexos en la pre-imagen, pero hay uno solo que contiene a \(\tilde
x_0\). Luego al hacer el homeomorfismo restringido de \(p\) con ese
conjunto, podemos utilizar la inversa para definir a \(\tilde f\) en el
intervalo inicial.

\subsection*{Pagina 25}
Para el punto final en \(\tilde X\), al hace su imagen bajo \(p\) le
corresponde otro punto en \(X\) sobre la curva \(f\) que tiene otro
conjunto que lo cubre por \(p\) y por tanto otra familia disconexa en
\(\tilde X\). Solo un conjunto de esta familia contiene al punto azul,
eligiendo esta podemos aprovechar otra vez el homeomorfismo restringido
de \(p\) para seguir definiendo a \(\tilde f\).

\subsection*{Pagina 26}
Ya sabemos que esto es un homomorfismo. La inyectividad equivale a
probar que si \([\gamma] \in \pi (\tilde X, \tilde x_0)\) cumple
\[ p_* ([\gamma]) = [e]\]
entonces \([\gamma] = [e] \in \pi (\tilde X, \tilde x_0)\). Esto se ve
notando que
\[ p_* ([\gamma]) = [p \circ \gamma] = [e]\]
dice que hay una homotopia \(H\) entre \(p \circ \gamma\) y \(e\). Esta
homotopia puede levantarse a \(\tilde X\). Donde sera una homotopia
entre
\[ \gamma \simeq k_{\tilde {x_0}} \]
mostrando asi lo que queremos

El teorema anterior no es tan util pues conocer el grupo fundamental de
\(\tilde X\) ya es dificil de por si. ¿Que proviene de estudiar
cubrimientos que tengan grupos fundamentales triviales?

\subsection*{Pagina 27}
un cubrimiento universal es aquel donde \(\tilde X\) es trivial. Este
nos permite definir la siguiente funcion biyectiva \(\phi\). Notar que
\(\phi\) esta bien definida pues dado cualquier representante de clase
de \([f]\) estos los relaciona una homotopia que fija puntos finales e
iniciales. Luego el valor de \(\tilde f (1)\) siempre es el mismo como
numero.

La inyectividad de esta funcion proviende de que si \([f] , [g]\) son
tales wque
\[ \phi ([f]) = \tilde f (1) = \tilde g (1) \phi ([g]) \]
y ademas \(\tilde f (0) = \tilde x_0 = \tilde g (0)\). Luego por ser
simplemente conexo, existe una homotopia \(\tilde H\) entre \(\tilde f\)
y \(\tilde g\). Luego
\[ p \circ \tilde H \]
es una homotpia entre \([f]\) y \([g]\) y por tanto \([f] = [g] \)

La sobreyectividad proviene de que para todo punto \(\tilde x_1 \in
\tilde X\) , por arco-conexidad siempre exsite un camino entre
\(\alpha\) entre \(\tilde x_0\) a \(\tilde x_1\) tal que
\[ \phi (p \circ \alpha) = \alpha (1) = \tilde x_1\]

Uno formalmente deberia de argumentar que \(\pi (G,p)\), el grupo de
automorfismos que actuan en la fibra \(p^{-1} (x_0)\) es isomorfo \(\pi
(X, x_0)\). Para esto uno estudia a \(\phi\) viendo como mueve los
puntos base de la fibra \(p^{-1} (x_0)\). Esto se hace en la tesis, sin
embargo podemos estudiar directamente las acciones mediante \(\phi\) y
proponer grupos fundamentales directamente si es que logramos dotar de
estructura de grupo a la fibra \(p^{-1} (x_0)\).

\subsection*{Pagina 28}
Con esto podemos plantearnos algunos ejemplos. Para ver el grupo
fundamental de \(\pi(S^1)\) notamos que \(R\) es un cubrimiento de \(S^1
\) mediante la funcion
\[ p (x) = (\cos 2 \pi x, \sin 2 \pi x) \]
y que \(R\) tiene grupo fundamental trivial porque es un espacio
convexo. Luego \(\phi\) es biyectiva. Debemos probar que es un
homomorfimos.
Sea \([\beta],[\delta] \in \pi (S^1)\), y sean \(\tilde
\beta , \tilde \delta\) sus levantamiento respectivos sobre \(\tilde x_0
\in R\) tales que
\[ \tilde \beta (1) = n \]
\[ \tilde \delta (1) = m \]

El levantamiento de \(\beta * \delta \) debe ser \(\tilde \beta
* \tilde {\tilde \delta}\) pues hay unicidad de levantamiento en la
primera parte y \(\tilde {\tilde \delta}\) el levantamiento de
\(\delta\) comenzando en \(n\).
\[ \tilde \beta * \tilde {\tilde \delta} (1) = n + \tilde \delta (1) =
  n + m = \tilde \beta (1) + \tilde \delta (1) \]
Por lo que se tiene que \(\phi \) es un homomorfismo biyectivo y el
grupo sobre \(R\) que tenemos es \(Z, +\)

Para ver el otro ejemplo, basta notar que \(RP^2\) se presenta como
\(S^2 \) con los inversos identificados. Luego \(p\) es el mapeo cuociente
y por tanto para cualquier \(x_0\) se tiene que
\[ p^{-1} (x_0) = \{x_0 , - x_0\}\]
tiene cardinalidad 2. El unico grupo de cardinalidad 2 es \(Z/2Z\).

\subsection*{Pagina 29}
Podemos definir una estructura categorica para un espacio \(X\) con
cubrimiento universa en donde hay presencia de un elemento inicial. Este
se mapea bajo un functor \(\pi\) al elemento inicial de
\(\mathscr{Grp}\) y estos cubrimientos pasan a ser inyecciones de grupos.

\subsection*{Pagina 30}
El teorema de \vank es el mas util a la hora de calcular. Dice que si un
espacio luce como un co-producto, su grupo fundamental esta relacionado
con el grupo libre generado por los grupos fundamentales de cada uno de
los sumandos. Dicho en el enunciado los morfismos \(\phi_1, \phi_2\)
seria morfismos a \(H\) el grupo libre generado por \(\pi (U,x_0) * \pi
(V,x_0)\). Recordemos que le grupo libre es el espacio generado por las
cadenas de estos grupos donde la identidad de grupo es la cadena vacia y
la yuxtaposicion es la operacion de grupo. Se pide de que los indices no
sean cosecutivamente iguales ya que se reduciria a la operacion de grupo
interna del indice.

Para probar este teorema, equivale a mostrar que para toda curva \([f]
\in \pi (X,x_0)\) se puede descomponer como
\[ g_1 * g_2 * g_3 \dots * g_n \]
con \(g_i\) totalmente contenido en \(\pi (U, x_0)\) o \(\pi (V, x_0)\).
Por el lema del numero de lebesgue existe una subdivision \(\{a_i ,
a_{i+1}\}\) de \(I\) la cual cumple que
\[ f \left( [a_i , a_{i+1}] \right) \subset U  \oplus f \left( [a_i ,
    a_{i+1}] \right) \subset V \]
y existe homeomorfismo \(l_i : [a_i , a_{i+1}] \to I \). Luego
\[ f_i = f \circ l_i \]
Pero estos \(f_i\) no son necesariamente caminos cerrados. Sus extremos
pertenecen a \(U \cap V\) el cual es arco-conexo, por lo que tomamos
funciones \(\alpha_i\) tales que
\[ \alpha_i (0) = x_0 , \alpha_i (1) = f(a_i)\]
Lueog
\[ g_i = \alpha_{i-1} * f_i * \overline{\alpha_i} \]
es un camino cerrado de \(\pi(U, x_0)\) o de \(\pi (V,x_0)\). Solo y se
cumple que
\[ [f] = [g_1] * [g_2] * \dots * [g_n] \]

Lo que se ha probado aqui es que \(\phi\) es sobreyectivo cuando \(H\)
es el grupo fundamental. Para formar un isomorfismo de grupo podriamos
utilizar el primer teorema de isomorfismos de grupos. La version general
de \vank caracteriza a este kernel diciendo que es conformoda por el
normalizador todas las cadenas
\[ [f]_{\pi (A,x_0)} \simeq [f]_{\pi (B, x_0)} \text{ si } [f] \in \pi
  (A \cap B, x_0) \]

La caracterizacion es que esta construccion es un push out. Mas
formalmente un push out de \(i_1, i_2\) son lo mosrifmos \(\phi_1,
\phi_2\) y \(H\) tal que para cualquier otro elemento con morfismo
\(j_1, j_2 , \pi (X,x_0)\). \vank se puede pensar como decirnos que los
push-outs de intersecciones arco-conexas se preservan bajo \(\pi\).

Veremos algunos ejemplos

\subsection*{Pagina 31}
\(S^1 \vee S^1\) es sencillo de ver pues es una aplicacion directa.
Tiene grupo fundamental Z * Z.

La botella de Klein la podemos estudiar con el teorema general. En el
diagrama estan los conjunto \(U\), \(V \) y \(U \cap V\). El grupo
fundamental de U es trivial. El grupo fundamental de \(V\) podemos
basarlo en un vertices el cual es unico al ser identificado, luego se ve
que este espacio esta generado por la curva \(b\) y la curva \(a\), es
decir
\[ \langle a,b \rangle \simeq Z * Z \]
Tenemos que estudiar el normalizador. Notar que \(\pi (U,V)\) es
homotopicamente equivalente a \(S^1\) luego su grupo fundamental es
\(Z\). LLamemos \([\gamma]\) a su unico elemento no trivial. Estudiamos
las inyeciones de
\[ i_{u,v} : \pi (U \cap V) \to \pi (U )\]
\[ i_{v,u} : \pi (U \cap V) \to \pi (V) \]
\[ i_{u,v} ([\gamma]) = e \]
\[ i_{v,u} ([\gamma]) = [\gamma]_{V} \]
Pero \([\gamma]\) en \(V\) es homotopicamente equivalente a \([a * b * a *
b^{-1}]\) por homotopia axial. Luego \(N = N \{[a][b][a][b^{-1}]\}\)
siendo este normalizador simplemente \(\{[a][b][a][b^{-1}]\}\) (esto
debe de verse por calculo). Luego \vank nos dice que el grupo
fundamental de este espacio es
\[ <a,b> / {abab^{-1} \simeq e}\]
\end{document}
