\documentclass[letterpaper]{article}

\usepackage[letterpaper]{geometry}
\usepackage[log-declarations=false]{xparse}
\usepackage[quiet]{fontspec}
\usepackage{amsmath}
\usepackage{amssymb}
\usepackage{mathtools}
\usepackage{amsfonts}
\usepackage{enumerate}
\usepackage{polyglossia}
\usepackage{unicode-math}

\newcommand{\vank}{\emph{Seifert-van Kampen} }

\setdefaultlanguage{spanish}

\begin{document}
\title{Discusión}
%Esto ira acompañando a la presentacion. Cada secciones sera una nueva
%pagina de la presentacion.

\section*{Pagina 2-7}
\subsection*{Pagina 2}
La motivación principal es definir una invariante topológica que
denominaremos grupos fundamental de un espacio topológico \(X\). Esta
tendrá una estructura algebraica (de grupo) y podrá reutilizar mucho del
trabajo de clasificación de grupos para clasificar espacio topológicos.

\paragraph{pausa 2}
La definición de esta invariante ocurre al estudiar los caminos
admisibles en un espacio \(X\) y las deformaciones que sean admisibles
entre estos dado el espacio \(X\).

\subsection*{Pagina 3}
Solo para afirmar el lenguaje, un camino sobre un espacio topológico
\(X\) es una función continua
\[ f : [0,1] \to X \]

\paragraph{Siguiente pausa}
Una homotopía es la formalización de las deformaciones admisible entre
dos funciones. Dicho formalmente \texttt{Mostrar definición en
  diapositiva} es dicha función continua donde el primer parámetro
evalúa en la curvas y el segundo corresponde a un grado de deformación

\subsection*{Pagina 4}
Si dos caminos son compatibles (ie tienen el mismo punto final e
inicial), entonces se puede definir el producto de ellos de la siguiente
forma \texttt{mostrar le definición de producto}

\paragraph{pausa 2}
Es de mayor interés los caminos cerrados (con mismo punto final e
inicial) pues el operación anterior esta bien definida siempre.

\subsection*{Pagina 5}
Para una homotopía se puede hacer un proceso análogo. Se utiliza la
misma notación.

\subsection*{Pagina 6}
Todo camino tiene un inversa, que ocurre naturalmente de recorren en
sentido contrario al original. Con esto definimos el siguiente conjunto
\texttt{mostrar \(\pi\)} el cual junto a la operación de producto
\texttt{mostrar \((*)\)} forman el grupo fundamental de \(X,x_0\). Notar
que esta operación sobre clases de equivalencias esta bien definida por
la definición de producto de homotopias.

\subsection*{Pagina 7}
Si queremos clasificar espacios topológicos, que el punto de origen
afecte al grupo fundamental es insatisfactorio. Por esto es que en
general esta invariante se usa principalmente para espacio arco-conexos.
De manera que siempre estamos a una conjugación de grupo de cualquier
otro punto base.

Aun en espacio arco-conexos vale la pena mantener la información del
punto de origen, pues existe una caracterización de grupo fundamental
Abeliano enumerando los posibles isomorfismos de grupo del grupo
fundamental con distintos puntos bases.

\subsection*{Pagina 8}
Es acaso el grupo fundamental una invariante? Esto equivale a mostrar
que si tenemos un homeomorfismo entre espacios, tambien tenemos un
isomorfismo entre grupos

\paragraph{Pausa 1}
Esto es cierto utilizando la siguiente construccion \texttt{mostrar}.
Este es un homorfismo de grupos pues
\[ f_* ([\alpha] * [\beta]) = [f \circ (\alpha * \beta) ] = [f \circ
  \alpha * f \circ \beta] = [f \circ \alpha ] * [f \circ \beta]\]
La inyectividad y sobreyectividad son consecuencia de la inyectividad y
sobreyectividad de \(f\).

\paragraph{Pausa 2}
Esta construccion es general, esto es un functor entre categorias
\(\mathscr{Top}_* \to \mathscr{Grp}\)

\subsection*{Pagina 9}
Un ejemplo intuitivo seria estudiar el espacio \(R^2\) menos un punto.
En este caso tomaremos \((1,1)\) pero puede ser cualquier otro. Se nos
ocurren tres grandes clases de caminos
\begin{itemize}
\item Aquellos que son homomorficos a \(k_0\). Estos son aquellos que no
  encierran a \((1,1)\) pues estan contenidos en un subespacio convexo y
  podemos utilizar la siguiente homotopia
  \[ L (t,\lambda) = \lambda f (t) - (1 - \lambda) k_{x_0} (t)\]
\item Aquellos que encierren a \((1,1)\) una vez. Los cuales pertenecen
  a su propia clase que denotaremos \(\mathbb F ^1\)
\item Aquellos que encienrren a \((1,1)\) mas veces. Siempre se pueden
  ver como homomorficos a \(\mathbb F * \dots * \mathbb F \)
\end{itemize}
Notamos el producto entre estas distintas clases equivale a levantar la
pontencia de la clase \(\mathbb F\), luego este grupo es isomorfo al
aditivo \((\mathbb Z , +)\). Esto es solo un ejemplo intuitivo,
necesitamos herramientas formales para determinar grupos fundamentales

\subsection*{Pagina 10}
\texttt{esta claro}

\subsection*{Pagina 11}
Una retraccion es la siguiente definicion \texttt{ver definicion}. Un
ejemplo de ella podria sea la funcion
\[ r(a) := \frac a {\lVert a \rVert} \]
entre \(R^2 - 0\) a \(S^1\)

De esto se tiene el siguiente teorema

\subsection*{Pagina 12}
Se tiene el siguiente teorema. \texttt{ver teorema}
La demostracion se basa en notar que
\[ r \circ j \equiv Id : A \to A \]
Luego \(j\) es inyectiva. Por tanto \(j_*\) tambien es inyectiva

¿La pregunta es si tenemos el sentido contrario? Necesitamos el
siguiente lema

\subsection*{Pagina 13}
Se necesita el siguiente lema \texttt{ver lema}. La demostracion se basa
en probar que
\[ [h \circ f] = [k \circ f ], \quad \forall f \in \pi (X, x_0)\]
Notar que dado que tenemos una homotopia \(H\) entre \(k , h\), se
construye la siguiente homotopia entre estos dos
\[ M(z,t) := H \circ (f(z), t) \]
(ie precomponer). La hipotesis de punto fijo es para obtener que sea una
arco homotopia pues

\[ \forall t \in I ,\ M(0,t) = H \left( f (0), t \right) = H (x_0, t)
    = y_0 = H \left( f(1) , t \right) = M (1, t) \]

\subsection*{Pagina 14}
generalizada. Ya tenemos la inyectividad de \(j_*\) por
\[ r \circ j = Id : A \to A \]
Estudiemos ahora
\[ j \circ r \simeq Id : X \to X\]
el cual deja fijo a cualquier punto \(a \in A\). Luego el lema anterior
nos dice que
\[ j_* \circ r_* = (j \circ r)_* = Id_* \]
por lo que \(j_*\) es sobreyectivo.

\subsection*{Pagina 15}
La generalizacion de esta tecnica lleva a la definicion de equivalencias
homotopicas \texttt{ver diapositiva}. ¿Como arreglamos el requerimiento
de un punto fijo para la homotopia de \(g \circ f \simeq Id\) ? Por la
arco conexidad existe un camino \(\alpha(t)\) entre
\[ g \circ f (x_0) \quad \text{ y } \quad x_0 \]
Pre-componemos por la funcion
\[ \hat \alpha ([f]) := [ \bar \alpha \circ f \circ \alpha ], \quad \forall
  [f] \pi (X, x_0) \]
Esto arregla pues uno puede probar que este \(\hat \alpha\) es un
isomorfismos de grupo (cambia el punto base en espacios arco-conexos).

\subsection*{Pagina 16}
Estos tres espacios no son homemorfos. El primero con el segundo son dos
bolas unidas por una linea y el segundo son dos bolas unidas por un
punto. Si fueran homemorfos, se tendria un homeomofismo entre una linea
y un punto. Analogamente para la tercera figura

Sin embargo estas son del mismo tipo homotopico pues todas son
homotopicamente equivalentes al disco \(D^2\) menos dos puntos. De hecho
las equivalencias homotopicas son las inyecciones y son homotopicas a la
identidas por el siguiente diagrama \texttt{ver diagrama}.

\subsection*{Pagina 17}
La banda de mobius se presenta como el cuadrado \(I^2\) con los lados
identificados en sentido inverso. Tomamo la recta vertical media y
notamos que siempre podemos hacer una homotipa a esta. Luego notamos que
esta recta (con los extremos identificados) es homemomorfa a \(S^1\).
Luego la banda de mobius tiene el mismo grupo fundamental de \(S^1\)

\subsection*{Pagina 18}
Con la nocion de equivalencias homotopicas podemos enfocarnos en las
familia de funciones continuas homotopicamente equivalentes que entre
dos espacios topologicos. Esto nos lleva a definir la categoria HoTop
que formaliza nuestros resultados \texttt{mostrar definicion en
  diapositiva}

\subsection*{Pagina 19}
En particular podemos ver a \(Top_*\) como una subcategoria mediante un
functor inyeccion adecuado. Notar que todo par de espacios topologicos
homeomorfos tienen un par de equivalencias en HoTop.

Se extiende el functor \(\pi\) a trabajar desde \(HoTop\). Todo par de
espacios homotopicos que tengan un par de equivalencias homotopicas,
cualquiera de estas equivalencias al ser levantas de \(f_*\) mediante
\(\pi\) se convierte en un isomorfismo de grupos.

\subsection*{Pagina 20}
Se tiene productos en esta categoria y el functor \(\pi\) preserva esta
estructura.

\subsection*{Pagina 21}
Tambien se tiene un co-producto dado por la siguiente construccion
\texttt{ver definicion}. La pregunta es si el functor preserva esta
estructura ? La respuesta es afirmativa, se vera mas adelante con \vank

Por ahora estudiaremos otros resultados que pueden ser resumidos en
terminos categoricos. Primero partiendo por cubrimientos y luego por el
teorema de \vank

\end{document}
