\documentclass[letterpaper]{article}

\usepackage[letterpaper]{geometry}
\usepackage[log-declarations=false]{xparse}
\usepackage[quiet]{fontspec}
\usepackage{amsmath}
\usepackage{amssymb}
\usepackage{mathtools}
\usepackage{amsfonts}
\usepackage{enumerate}
\usepackage{polyglossia}

\setdefaultlanguage{spanish}

\begin{document}
\title{Discusión}
%Esto ira acompañando a la presentacion. Cada secciones sera una nueva
%pagina de la presentacion.

\section*{Pagina 2-7}
\subsection*{Pagina 2}
La motivación principal es definir una invariante topológica que
denominaremos grupos fundamental de un espacio topológico \(X\). Esta
tendrá una estructura algebraica (de grupo) y podrá reutilizar mucho del
trabajo de clasificación de grupos para clasificar espacio topológicos.

\paragraph{pausa 2}
La definición de esta invariante ocurre al estudiar los caminos
admisibles en un espacio \(X\) y las deformaciones que sean admisibles
entre estos dado el espacio \(X\).

\subsection*{Pagina 3}
Solo para afirmar el lenguaje, un camino sobre un espacio topológico
\(X\) es una función continua
\[ f : [0,1] \to X \]

\paragraph{Siguiente pausa}
Una homotopía es la formalización de las deformaciones admisible entre
dos funciones. Dicho formalmente \texttt{Mostrar definición en
  diapositiva} es dicha función continua donde el primer parámetro
evalúa en la curvas y el segundo corresponde a un grado de deformación

\subsection*{Pagina 5-6}
Si dos caminos son compatibles (ie tienen el mismo punto final e
inicial), entonces se puede definir el producto de ellos de la siguiente
forma \texttt{mostrar le definición de producto}

\paragraph{pausa 2}
Es de mayor interés los caminos cerrados (con mismo punto final e
inicial) pues el operación anterior esta bien definida siempre.

\paragraph{pausa 3}
Para una homotopía se puede hacer un proceso análogo. Se utiliza la
misma notación.

\subsection*{Pagina 6}
Todo camino tiene un inversa, que ocurre naturalmente de recorren en
sentido contrario al original. Con esto definimos el siguiente conjunto
\texttt{mostrar \(\pi\)} el cual junto a la operación de producto
\texttt{mostrar \((*)\)} forman el grupo fundamental de \(X,x_0\). Notar
que esta operación sobre clases de equivalencias esta bien definida por
la definición de producto de homotopias.

\subsection*{Pagina 7}
Si queremos clasificar espacios topológicos, que el punto de origen
afecte al grupo fundamental es insatisfactorio. Por esto es que en
general esta invariante se usa principalmente para espacio arco-conexos.
De manera que siempre estamos a una conjugación de grupo de cualquier
otro punto base.

Aun en espacio arco-conexos vale la pena mantener la información del
punto de origen, pues existe una caracterización de grupo fundamental
Abeliano enumerando los posibles isomorfismos de grupo del grupo
fundamental con distintos puntos bases.

\end{document}
